\documentclass[12pt]{article}
\usepackage[usenames,dvipsnames]{color}
\usepackage{listings}
\usepackage{graphicx}
\usepackage{fancyhdr}
\usepackage{framed}
\usepackage[T1]{fontenc}
\usepackage[toc,page]{appendix}
\usepackage[utf8]{inputenc}
\usepackage[brazil]{babel}
\usepackage{fancyvrb}
\usepackage[hmargin=2cm,vmargin=2cm]{geometry}
\usepackage{lastpage}
\usepackage{makeidx}
\pagestyle{fancy}

% cabecalho e rodapé
\setlength{\headheight}{120pt}
\setlength{\textheight}{550pt}
\renewcommand{\headrulewidth}{0pt}
\lhead{\includegraphics[scale=0.03]{brasao.png}}
\rhead{\includegraphics[scale=0.4]{logo-pnud.png}}
\cfoot{\textbf{\ProjectCode\ - Inovando a democracia participativa}}
\rfoot{\thepage}

\hyphenation{par-ti-ci-pa-ção}
\bibliographystyle{ieeetr}

% definições sobre o autor e o produto
\newcommand{\MyName}{Joenio Marques da Costa}
\newcommand{\MySurnameForename}{Costa, Joenio}
\newcommand{\SupervisorName}{Gabriella Vieira Oliveira Gonçalves}
\newcommand{\MyEmail}{joenio@colivre.coop.br}
\newcommand{\ContractNumber}{2013/000564}
\newcommand{\ContractYear}{2013}
\newcommand{\ProjectCode}{Projeto BRA/12/018}
\newcommand{\NomeSecretaria}{Secretaria Geral da Presidência da República}
\newcommand{\SiglaSecretaria}{SG/PR}
\newcommand{\ProductNumber}{04}
\newcommand{\ProductTitle}{Proposta de aperfeiçoamento para aplicativos de
  participação social}
\newcommand{\ProductSubtitle}{Aperfeiçoamento das trilhas de participação e
  interface de gestão dos aplicativos de participação social}
\newcommand{\ProductDescription}{"Documento com proposta de aperfeiçoamento
  para os aplicativos das trilhas de Participação Social, com propostas de
  funcionalidades e exemplos de códigos."
}
\newcommand{\ProductValue}{R\$ 10.800,00 (dez mil e oitocentos reais)}
\newcommand{\ObjetoContratacao}{"Construção dos códigos para comunidades e
  aplicativos do portal da participação social."
}
\newcommand{\DataEntrega}{?? Setembro de 2014}
\newcommand{\PalavrasChave}{palavra1, palavra2, palavra3, etc}

% lista de abreviações
\makeindex

\begin{document}

\newgeometry{hmargin=3cm,vmargin=1.5cm}
\begin{center}
\thispagestyle{empty}
{\color{MidnightBlue}

\includegraphics[scale=0.9]{logo-pnud.png}

\vspace{4cm}

{\bf \large \ProjectCode\ - Desenvolvimento de Metodologias
de Articulação e Gestão de Políticas Públicas para Promoção da Democracia
Participativa}

\vspace{1.5cm}

{\bf \large Produto \ProductNumber\ -\ \ProductDescription}

\vspace{1.5cm}

\ProductSubtitle

\vspace{2.5cm}

\MyName

\vspace{1.5cm}

}

\includegraphics[scale=0.04]{brasao.png} \\
{\bf \small \NomeSecretaria}

\end{center}
\restoregeometry
\newpage

\newgeometry{hmargin=3cm,vmargin=1.5cm}
\addtolength{\topmargin}{2.5cm}
\thispagestyle{empty}

{\bf \ProjectCode \ -} \ProductDescription

\vspace{2.5cm}

\begin{minipage}{\textwidth}
  {\bf Consultor(a): \MyName}
  \newline
  {\bf Contrato nº: \ContractNumber}
  \newline
  {\bf Produto / nº: \ProductNumber}
\end{minipage}

\vspace{2cm}

{\bf Assinatura do Consultor(a)}

\begin{framed}
\noindent Local e data: Brasília/DF, \line(1,0){20} \ de \line(1,0){100} \ de 2014
\newline
\newline
Assinatura do Consultor(a): \line(1,0){300}
\end{framed}

\vspace{1cm}

{\bf Assinatura do Supervisor(a)}

\begin{framed}
\noindent Atesto que os serviços foram prestados conforme estabelecido no
Contrato de Consultoria.
\newline
\newline
Local e data: Brasília/DF, \line(1,0){20} \ de \line(1,0){100} \ de 2014
\newline
\newline
Assinatura e Carimbo: \line(1,0){300}
\end{framed}

\vspace{1.5cm}

\begin{center}
\includegraphics[scale=0.04]{brasao.png} \\
{\bf \small \NomeSecretaria}
\end{center}

\restoregeometry
\newpage

\newgeometry{hmargin=3cm,vmargin=1.5cm}
\addtolength{\topmargin}{2.5cm}
\thispagestyle{empty}


\makebox{

\MySurnameForename \\
	\ProductTitle: \ProductSubtitle\ / \ContractYear.
\newline
	Total de folhas: \pageref{LastPage}

	Supervisor:
	Secretaria (Indicar SE/SNAS/SNARPS/SNJ etc.)
Secretaria-Geral da Presidência da República
	Palavras-chave:
}


{\raggedright \includegraphics{licenca-cc-by-nc.png} \ Esta obra é licenciada sob
uma licença Creative Commons - Atribuição-NãoComercial. 4.0 Internacional.}

\restoregeometry
\newpage

\tableofcontents
\newpage

\begin{abstract}
escrever resumo aqui... \\

{\bf Palavras-chave:} \PalavrasChave.
\end{abstract}
\newpage

\section{Introdução}

Em consonância com os objetivos e cronograma previsto no âmbito do
projeto BRA/12/018:
\textbf{Desenvolvimento de Metodologias de Articulação e Gestão de
Políticas Públicas para Promoção da Democracia Participativa},
firmado entre a Secretaria-Geral da Presidência da República
(SG/PR) e o Programa das Nações Unidas para o Desenvolvimento (PNUD),
o presente documento apresenta \ProductDescription.

Essa proposta está configurada como produto \ProductNumber~da consultoria técnica
para especificação da construção dos códigos das metodologias de
organização da informação e interação participativa do portal de
participação social.

\section{Desenvolvimento}

O Participa.br é a Plataforma Federal da Participação Social. Trata-se de mais
um espaço para participação social no Brasil, escuta e diálogo entre o Governo
Federal e a Sociedade Civil. 

A plataforma, totalmente desenvolvida em software livre, tem como missão
desenvolver práticas inovadoras de participação via internet e oferta de
espaços de manifestação e debate para qualquer cidadão ou organização, com o
intuito de construir políticas públicas cada vez mais eficazes e efetivas.

O Participa.br é desenvolvido sob a plataforma para redes sociais Noosfero.

\subsection{O Noosfero}

O Noosfero é uma plataforma web livre para redes sociais e de economia
solidária que possui as funcionalidades de Blog, e-Portfolios, CMS, RSS,
discussão temática, agenda de eventos e inteligência econômica colaborativa
num mesmo sistema! O Noosfero utiliza a linguagem de programação Ruby com
framework Rails e, portanto, suporta bancos de dados, PostgreSQL, MySQL,
SQLite entre outros.

Noosfero é um Software Livre e licenciado sob a GNU Affero General Public
License (AGPL), versão 3\cite{wikipediaSingleSignOn}.

\subsection{Aperfeiçoamento para os aplicativos de Participação Social}

\subsubsection{Observatório do Participa.br}

O Participa.br organiza seus debates em torno de comunidades temáticas criadas
a partir do interesse da sociedade ou governo. A gestão das comunidades é
conjunta. A construção de um processo participativo dentro de uma comunidade
ocorre através de criação de diversos tipos de conteúdos e ferramentas
digitais de participação. Estes conteúdos e ferramentas podem ser criados por
qualquer usuário do ambiente participativo e possuem em comum 2 informações de
vital importancia para dar sustentação ao observatório aqui proposto, são
elas: tags e categorias (ver Figura~\ref{categorias-tags}), elas dão um viés
estrutural a todo conteúdo e ferramenta de participação criada no
Participa.br.

\begin{figure}[h]
\center
\includegraphics[scale=0.6]{categorias-tags.png}
\caption{Definição de categoria e tag na edição de conteúdo}
\label{categorias-tags}
\end{figure}

O observatório do Participa.br dará aos usuários uma forma de acompanhar temas
de interesse através da seleção de tags (ver Figura~\ref{observatorio-tags}) e
categorias (ver Figura~\ref{observatorio-categorias}), isto proporcionará ao
usuário uma visão personalizada de tudo que acontece no Participa.br (ver
Figura~\ref{observatorio-busca}).

\begin{figure}[h]
\center
\includegraphics[scale=0.4]{observatorio-tags.png}
\caption{Seleção de tags para acompanhar no observatório}
\label{observatorio-tags}
\end{figure}

\begin{figure}[h]
\center
\includegraphics[scale=0.4]{observatorio-categorias.png}
\caption{Seleção de categorias para acompanhar no observatório}
\label{observatorio-categorias}
\end{figure}

\begin{figure}[h]
\center
\includegraphics[scale=0.25]{observatorio-busca.png}
\caption{Observatório - Página de busca principal}
\label{observatorio-busca}
\end{figure}

Será possível ainda, acompanhar o observatório a partir de leitor de feeds
externos, pois será disponibilizado um FEED RSS a partir dos conteúdos do
observatório conforme pode ser visto na Figura~\ref{observatorio-wireframe}.

\begin{figure}[h]
\center
\includegraphics[scale=0.5]{observatorio-wireframe.png}
\caption{Observatório - Página de busca com opções de filtros}
\label{observatorio-wireframe}
\end{figure}

Para possibilitar a implementacao do observatorio sera preciso:
\begin{itemize}
  \item Criar uma forma de visuzlizacao de todas as categorias assim como há para tags
  \item Permitir aos usuarios seguir tags e categorias
  \item Implementar uma nova funcionalidade que permita ao usuário visualizar conteúdos com base nas tags e categorias que está seguindo
\end{itemize}

Issues relacionadas ao observatório:
\begin{itemize}
  \item https://gitlab.com/participa/noosfero/issues/66
\end{itemize}

Mais wireframes sobre esta feature em:

\begin{itemize}
  \item http://sgpr.faracy.com.br/wireframe/Meu\%20Observatorio.html
  \item http://sgpr.faracy.com.br/wireframe/Lista\%20de\%20Temas\%2029-09.html
  \item http://sgpr.faracy.com.br/wireframe/Nuvem\%20de\%20Tags\%2027-09.html
\end{itemize}

\subsubsection{Painel gestor das ferramentas de consulta}

Proposta de um painel administrativo e de sistematização de consultas com base
nas ferramentas de comentários por trecho e por parágrafo usando etiquetas e
status definidos dinamicamente via interface administrativa.

Esta proposta de ferramenta de sistematização prevê o uso do plugin Noosfero
CommentClassification\cite{commentClassificationPlugin}, este plugin
possibilita classificar comentários através de etiquetas e status, ele foi
desenvolvido por Daniela Feitosa como requisito para o Produto04, a partir
deste plugin os usuários podem indicar semanticamente qual o significado do
seu comentário no momento em que o faz, como pode ser visto na
Figura~\ref{etiqueta}.

\begin{figure}[h]
\center
\includegraphics[scale=0.5]{etiqueta.png}
\caption{Etiquetas de classificação para comentários}
\label{etiqueta}
\end{figure}

O administrador da consulta pode gerenciar as etiquetas e os status que serão
posteriormente disponibilizados para os usuários como pode ser visto nas
Figuras \ref{manage-labels} e \ref{manage-status}. Os status são utilizados
pelos avaliadores enquanto analisam os comentários e poderão definir por
exemplo que um comentário faz sentido ou não.

\begin{figure}[h]
\center
\includegraphics[scale=0.5]{manage-labels.png}
\caption{Gerenciamento de labels do plugin CommentClassification}
\label{manage-labels}
\end{figure}

\begin{figure}[h]
\center
\includegraphics[scale=0.5]{manage-status.png}
\caption{Gerenciamento de status do plugin CommentClassification}
\label{manage-status}
\end{figure}

Desta forma o administrador da consulta pode pré-definir etiquetas que serão
utilizadas pelos usuário no momento do comentário, o administrador pode por
exemplo criar etiquetas que indiquem:

\begin{itemize}
  \item Adição ou remoção de texto
  \item Concordancia ou discordancia
  \item Opinião sobre qualidade do texto (bom, ruim, etc)
  \item Dentre outros, a criação de etiquetas é livre e o admin pode criar quantas quiser
\end{itemize}

Cada label pode ter uma cor, essa cor será usada na visualização do
comentário.

O administrador da consulta pode a partir do painel de controle de uma
comunidade como na Figura~\ref{control-panel} gerenciar os comentários e
definir status para eles, estes status são pré-definidos através da
configuração do plugin.

\begin{figure}[h]
\center
\includegraphics[scale=0.5]{control-panel.png}
\caption{Imagem no painel de controle da comunidade}
\label{control-panel}
\end{figure}

Além de permitir gerenciar cada comentário de forma individual o painel de
controle de gestão deve disponibilizar uma opção de exportação em CSV dos
trechos, comentários, status e justificativas de forma que seja possível
analisar tais dados externamente a partir de ferramentas de análise.

Como um trecho pode ter vários comentários e cada comentário pode ter várias
sugestões de justificativas e alteração de texto, vê se dessa forma está bem
organizado. (colocar imagem de comentários aninhados)

\begin{figure}[h]
\center
\includegraphics[scale=0.5]{support-on-comment.png}
\caption{Apoio/aprovo este comentario}
\label{support-on-comment}
\end{figure}

%http://noosfero.org/Development/ActionItem2520

Pode se fazer necessário a união de duas unidades comentáveis e para isto
basta englobar os trechos e permitir que comentários que faziam parte de 2
trechos distintos sejam exibidos de forma aglutinada, assim... Assim todos os
comentários são listados de forma unificada para os dois parágrafos e somente
será exibido uma unidade comentável.

O plugin permite marcar identificadores na tag comentável; através destes
identificadores filtramos informações e acompanhamos uma consulta, a criação
desses identificadores tem sido de livre uso para o administrador e geralmente
esta associada e ou referenciada ao trecho comentável, exemplo: p1 = paragrafo
1.. art2 = artigo 2... cap1 = capitulo 1.

Esta funcionalidade de sistematização deve trazer o recurso de gerar novas
versões do documento (em caso de consultas com texto base) a partir da
aprovação ou rejeição de comentários (através das etiquetas), com isso o admin
conseguirá facilmente consolidar as contribuições dos usuários na consulta
pública.

Tanto os status quanto os labels podem ser "desabilitados", para que as
pessoas não possam escolher.

E também tem o "Add status", que só é visto pelas pessoas com permissão de "moderar comentários"

Deve ser possível ter uma tela para mostrar as estatísticas na própria
ferramenta: qtos querem adicionar, suprimir, questionamentos etc.
Um avaliador deve ter disponível um botão para poder concordar com a
justificativa de outro avaliador. Ao analisar um comentário é necessário
mostrar o trecho que está sendo comentado. Gerar um .odt com as informações
aglutinadas.

===========================================================

É importante que a interface indique de forma clara os comentários que já passaram por "colocação de status" incentivando a equipe de sistematização a responder todos. Podia ser uma "mascara" de cor em cima do avatar do comentador, ou outra solução que possa ser bem visível. 

Para exibição dos status, justificativas e novas redações acredito que a melhor solução seja mesmo aquelade duas colunas: 1) Coluna da direita a minuta com os blocos e trédis de comentários 2) Coluna daesquerda as "trédis de status/históricos" do comentário em "foco".
 
 - Ícone no painel de controle para visualizar a classificação dos comentários
 - Permitir que uma comunidade desative a função no perfil dela, mesmo estando ativo no ambiente
 - Gerar um .odt com as informações

https://gitlab.com/participa/noosfero/issues/33

================================

O Participa.br conta com inúmeras ferramentas de consulta pública e
participação popular, como:

* Ferramenta para construçao e debates de propostas (referencia produto3 joenio)
* Ferramenta de comentários por parágrafos (Comment Group Plugin)
* Comentários em artigos e comentário por trecho
* Votação de artigos/conteúdos (Vote Plugin)

%=============================================
%Criar issue para "citar" pessoas nos comentários, conteúdo também, exemplo facebook.
%Mostrar um link e balãozinho nestas citações, criar um alerta para a pessoa que foi citada.
%===============================================================

\section{Conclusão}

Neste documento foi apresentado um \ProductDescription

Lembramos que para tornar o Portal de Consulta Pública realmente um canal de
consulta e participação popular na discussão e na definição da agenda
prioritária do país, é necessário que além de documentação faça-se um esforço
de movimentar as pessoar fora do ambiente virtual, para que haja um
engajamento no uso e contribuição deste projeto de forma consistente e perene.

\newpage
\bibliography{bibliografia}
\newpage
\listoffigures
\newpage
\printindex
\newpage
\definecolor{lightgrey}{rgb}{0.95,0.95,0.95}
\lstset{language=Ruby,basicstyle=\small\ttfamily,backgroundcolor=\color{lightgrey}}

\section{Anexos}

\subsection{Exemplo de código do plugin Noosfero para pairwise}

\lstinputlisting{pairwise_content.rb}

\subsection{Exemplo de código XML retornado pelo pairwise-api}

\begin{lstlisting}
<?xml version="1.0" encoding="UTF-8"?>
<prompt>
  <created-at type="datetime">2010-07-01T23:48:01+00:00</created-at>
  <id type="integer">1</id>
  <left-choice-id type="integer">10</left-choice-id>
  <question-id type="integer">7</question-id>
  <right-choice-id type="integer">9</right-choice-id>
  <tracking nil="true"></tracking>
  <updated-at type="datetime">2010-07-01T23:48:01+00:00</updated-at>
  <votes-count type="integer">0</votes-count>
  <left-choice-text>bar</left-choice-text>
  <right-choice-text>foo</right-choice-text>
</prompt>
\end{lstlisting}

%\appendix
%\appendixpage
%\section{Foo bar}
\label{foobar}

%\lstinputlisting{observatorio.rb}


\end{document}
