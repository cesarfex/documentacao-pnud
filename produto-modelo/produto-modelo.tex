\documentclass[12pt]{article}
\usepackage[usenames,dvipsnames]{color}
\usepackage{listings}
\usepackage{graphicx}
\usepackage{fancyhdr}
\usepackage{framed}
\usepackage[T1]{fontenc}
\usepackage[toc,page]{appendix}
\usepackage[utf8]{inputenc}
\usepackage[brazil]{babel}
\usepackage{fancyvrb}
\pagestyle{fancy}
%\usepackage[scaled=0.92]{helvet}
%\normalfont
\usepackage[hmargin=2cm,vmargin=2cm]{geometry}
%\setlength{\headsep}{30pt}
%\setlength{\textheight}{550pt}
\usepackage{lastpage}

% cabecalho
\setlength{\headheight}{120pt}
\renewcommand{\headrulewidth}{0pt}
\lhead{\includegraphics[scale=0.03]{brasao.png}}
\rhead{\includegraphics[scale=0.4]{logo-pnud.png}}
\cfoot{}

\rfoot{\thepage}

\hyphenation{par-ti-ci-pa-ção}
\bibliographystyle{ieeetr}

\newcommand{\MyName}{Joenio Marques da Costa}
\newcommand{\MySurnameForename}{Costa, Joenio}
\newcommand{\SupervisorName}{Ricardo Poppi???}
\newcommand{\MyEmail}{joenio@colivre.coop.br}
\newcommand{\ContractNumber}{2013/000564}
\newcommand{\ContractYear}{2014 ???}
\newcommand{\ProjectCode}{Projeto BRA/12/018}
\newcommand{\NomeSecretaria}{Secretaria Geral da Presidência da República}
\newcommand{\SiglaSecretaria}{SG/PR}
\newcommand{\ProductNumber}{03}
\newcommand{\ProductTitle}{Título do Produto}
\newcommand{\ProductSubtitle}{Subtítulo do Produto}
\newcommand{\ProductDescription}{"Documento com proposta para desenvolvimento
        do código de 2 aplicativos para as trilhas de participação social
        priorizados nas reuniões do projeto contendo exemplos e códigos."
}
\newcommand{\ProductValue}{R\$ xx,xx (por extenso)}
\newcommand{\ObjetoContratacao}{"Transcrever o objeto da contratação conforme
  Termo de Referência e/ou contrato."
}
\newcommand{\DataEntrega}{?? Julho de 2014}
\newcommand{\PalavrasChave}{palavra1, palavra2, palavra3, etc}





\begin{document}

%\lstset{language=Ruby}
%\definecolor{light-gray}{gray}{0.95}
%\lstdefinestyle{codeFrame}{backgroundcolor=\color{light-gray},frame=lines}

\newgeometry{hmargin=3cm,vmargin=1.5cm}
\begin{center}
\thispagestyle{empty}
{\color{MidnightBlue}

\includegraphics[scale=0.9]{logo-pnud.png}

\vspace{4cm}

{\bf \large \ProjectCode\ - Desenvolvimento de Metodologias
de Articulação e Gestão de Políticas Públicas para Promoção da Democracia
Participativa}

\vspace{1.5cm}

{\bf \large Produto \ProductNumber\ -\ \ProductDescription}

\vspace{1.5cm}

\ProductSubtitle

\vspace{2.5cm}

\MyName

\vspace{1.5cm}

}

\includegraphics[scale=0.04]{brasao.png} \\
{\bf \small \NomeSecretaria}

\end{center}
\restoregeometry
\newpage

\newgeometry{hmargin=3cm,vmargin=1.5cm}
\addtolength{\topmargin}{2.5cm}
\thispagestyle{empty}

{\bf \ProjectCode \ -} \ProductDescription

\vspace{2.5cm}

\begin{minipage}{\textwidth}
  {\bf Consultor(a): \MyName}
  \newline
  {\bf Contrato nº: \ContractNumber}
  \newline
  {\bf Produto / nº: \ProductNumber}
\end{minipage}

\vspace{2cm}

{\bf Assinatura do Consultor(a)}

\begin{framed}
\noindent Local e data: Brasília/DF, \line(1,0){20} \ de \line(1,0){100} \ de 2014
\newline
\newline
Assinatura do Consultor(a): \line(1,0){300}
\end{framed}

\vspace{1cm}

{\bf Assinatura do Supervisor(a)}

\begin{framed}
\noindent Atesto que os serviços foram prestados conforme estabelecido no
Contrato de Consultoria.
\newline
\newline
Local e data: Brasília/DF, \line(1,0){20} \ de \line(1,0){100} \ de 2014
\newline
\newline
Assinatura e Carimbo: \line(1,0){300}
\end{framed}

\vspace{1.5cm}

\begin{center}
\includegraphics[scale=0.04]{brasao.png} \\
{\bf \small \NomeSecretaria}
\end{center}

\restoregeometry
\newpage

\newgeometry{hmargin=3cm,vmargin=1.5cm}
\addtolength{\topmargin}{2.5cm}
\thispagestyle{empty}


\makebox{

\MySurnameForename \\
	\ProductTitle: \ProductSubtitle\ / \ContractYear.
\newline
	Total de folhas: \pageref{LastPage}

	Supervisor:
	Secretaria (Indicar SE/SNAS/SNARPS/SNJ etc.)
Secretaria-Geral da Presidência da República
	Palavras-chave:
}


{\raggedright \includegraphics{licenca-cc-by-nc.png} \ Esta obra é licenciada sob
uma licença Creative Commons - Atribuição-NãoComercial. 4.0 Internacional.}

\restoregeometry
\newpage





\tableofcontents

\clearpage

\listoffigures
\clearpage

\begin{abstract}
The software architecture of a computer program represents the basic structure
and the externally visible relationships of those components. The documentation
of this architecture is of great importance to monitor and evaluate the
evolution of a system and to measure their attributes of modularity as coupling
and cohesion. A tool capable of extracting this documentation automatically
enables monitor and measure these attributes during the evolution of a software
project. This paper presents the implementation of a tool for automatic
extraction of information of dependence between modules for programs written in
C/C++ with focus on attributes of modularity to monitor the evolution of the
system. At the end of this paper is done an evaluation of this
tool through a case study.

{\bf Keywords:} software engineering, metrics, software architecture, coupling,
cohesion, C, C++, depedency.
\end{abstract}

\section{Intruducao}

Em consonância com os objetivos e cronograma previsto no âmbito do
projeto BRA/12/018:
\textbf{Desenvolvimento de Metodologias de Articulação e Gestão de
Políticas Públicas para Promoção da Democracia Participativa},
firmado entre a Secretaria-Geral da Presidência da República
(SG/PR) e o Programa das Nações Unidas para o Desenvolvimento (PNUD),
o presente documento apresenta \ProductDescription.

Essa proposta está configurada como produto \ProductNumber~da consultoria técnica
para especificação da construção dos códigos das metodologias de
organização da informação e interação participativa do portal de
participação social.

\section{O Participa.br}

O Participa.br é a Plataforma Federal da Participação Social. Trata-se de mais
um espaço para participação social no Brasil, escuta e diálogo entre o Governo
Federal e a Sociedade Civil. 

A plataforma, totalmente desenvolvida em software livre, tem como missão
desenvolver práticas inovadoras de participação via internet e oferta de
espaços de manifestação e debate para qualquer cidadão ou organização, com o
intuito de construir políticas públicas cada vez mais eficazes e efetivas.

O Participa.br é desenvolvido sob a plataforma para redes sociais Noosfero.

\section{O Noosfero}

O Noosfero é uma plataforma web livre para redes sociais e de economia
solidária que possui as funcionalidades de Blog, e-Portfolios, CMS, RSS,
discussão temática, agenda de eventos e inteligência econômica colaborativa
num mesmo sistema! O Noosfero utiliza a linguagem de programação Ruby com
framework Rails e, portanto, suporta bancos de dados, PostgreSQL, MySQL,
SQLite entre outros.

Noosfero é um Software Livre e licenciado sob a GNU Affero General Public
License (AGPL), versão 3. 

\section{Aplicativos para trilhas de participação}

O Participa.br utiliza algumas ferramentas como mecanismo de participação,
essas ferramentas permitem aos usuários da rede contribuir através de
comentários, votos ou diretamente sugerindo textos e conteúdos.

Estas ferramentas são organizadas em trilhas de participação, uma trilha
pode utilizar uma ou mais ferramentas e podem ter datas de de início e fim de
participação, dando assim um fluxo para a participação social.

Nas próximas seções será apresentado 2 propostas de ferramentas ou aplicativos
para trilha de participação.

\subsection{Comentários por parágrafo}

Comentário por parágrafo o usuário pode comentar em parágrafos específicos de
um texto sugerindo alterações ou correções, ou mesmo tirando dúvidas.

\subsection{Pairwise}

Votação em pares...

\section{Considerações finais}

Neste documento foi apresentado um \ProductDescription

Lembramos que para tornar o Portal de Consulta Pública realmente um canal de
consulta e participação popular na discussão e na definição da agenda
prioritária do país, é necessário que além de documentação faça-se um esforço
de movimentar as pessoar fora do ambiente virtual, para que haja um
engajamento no uso e contribuição deste projeto de forma consistente e perene.

%\bibliography{bibliografia}

\vspace{1cm}

Sem mais nada a acrescentar, coloco-me à disposição.

\vspace{1cm}

\begin{minipage}{\textwidth}
  Brasília/DF, \DataEntrega\\[1cm]
  \textbf{\MyName}\\
  \small Consultor do PNUD
\end{minipage}

\end{document}
