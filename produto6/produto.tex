\documentclass[12pt]{article}
\usepackage[usenames,dvipsnames]{color}
\usepackage{listings}
\usepackage{graphicx}
\usepackage{fancyhdr}
\usepackage{framed}
\usepackage[T1]{fontenc}
\usepackage[toc,page]{appendix}
\usepackage[utf8]{inputenc}
\usepackage[brazil]{babel}
\usepackage{fancyvrb}
\usepackage[hmargin=2cm,vmargin=2cm]{geometry}
\usepackage{lastpage}
\usepackage{makeidx}
\pagestyle{fancy}

% cabecalho e rodapé
\setlength{\headheight}{120pt}
\setlength{\textheight}{550pt}
\renewcommand{\headrulewidth}{0pt}
\lhead{\includegraphics[scale=0.03]{brasao.png}}
\rhead{\includegraphics[scale=0.4]{logo-pnud.png}}
\cfoot{\textbf{\ProjectCode\ - Inovando a democracia participativa}}
\rfoot{\thepage}

\hyphenation{par-ti-ci-pa-ção}
\bibliographystyle{ieeetr}

% definições sobre o autor e o produto
\newcommand{\MyName}{Joenio Marques da Costa}
\newcommand{\MySurnameForename}{Costa, Joenio}
\newcommand{\SupervisorName}{Ricardo Augusto Poppi Martins}
\newcommand{\MyEmail}{joenio@colivre.coop.br}
\newcommand{\ContractNumber}{2013/000564}
\newcommand{\ContractYear}{2013}
\newcommand{\ProjectCode}{Projeto BRA/12/018}
\newcommand{\NomeSecretaria}{Secretaria Geral da Presidência da República}
\newcommand{\SiglaSecretaria}{SG/PR}
\newcommand{\ProductNumber}{06}
\newcommand{\ProductTitle}{Proposta de aperfeiçoamento para aplicativos de
  participação social}
\newcommand{\ProductSubtitle}{Aperfeiçoamento das trilhas de participação e
  interface de gestão dos aplicativos de participação social}
\newcommand{\ProductDescription}{"Documento com análise de protocolos,
  arquiteturas e sistemas de federação de conteúdos para ambientes de redes
  Sociais com estratégia de implantação considerando os sites parceiros e
  contendo propostas de códigos. Inclui especificações e códigos para conexão
  de contas e trocas de postagens do portal com redes sociais proprietárias."
}
\newcommand{\ProductValue}{R\$ 14.400,00 (quatorze mil e quatrocentos reais)}
\newcommand{\ObjetoContratacao}{"Construção dos códigos para comunidades e
  aplicativos do portal da participação social."
}
\newcommand{\DataEntrega}{?? Novembro de 2014}
\newcommand{\PalavrasChave}{1, 2, 3, ...}

% lista de abreviações
\makeindex

\begin{document}

\newgeometry{hmargin=3cm,vmargin=1.5cm}
\addtolength{\topmargin}{2.5cm}
\thispagestyle{empty}

{\bf \ProjectCode \ -} \ProductDescription

\vspace{2.5cm}

\begin{minipage}{\textwidth}
  {\bf Consultor(a): \MyName}
  \newline
  {\bf Contrato nº: \ContractNumber}
  \newline
  {\bf Produto / nº: \ProductNumber}
\end{minipage}

\vspace{2cm}

{\bf Assinatura do Consultor(a)}

\begin{framed}
\noindent Local e data: Brasília/DF, \line(1,0){20} \ de \line(1,0){100} \ de 2014
\newline
\newline
Assinatura do Consultor(a): \line(1,0){300}
\end{framed}

\vspace{1cm}

{\bf Assinatura do Supervisor(a)}

\begin{framed}
\noindent Atesto que os serviços foram prestados conforme estabelecido no
Contrato de Consultoria.
\newline
\newline
Local e data: Brasília/DF, \line(1,0){20} \ de \line(1,0){100} \ de 2014
\newline
\newline
Assinatura e Carimbo: \line(1,0){300}
\end{framed}

\vspace{1.5cm}

\begin{center}
\includegraphics[scale=0.04]{brasao.png} \\
{\bf \small \NomeSecretaria}
\end{center}

\restoregeometry
\newpage

\newgeometry{hmargin=3cm,vmargin=1.5cm}
\begin{center}
\thispagestyle{empty}
{\color{MidnightBlue}

\includegraphics[scale=0.9]{logo-pnud.png}

\vspace{4cm}

{\bf \large \ProjectCode\ - Desenvolvimento de Metodologias
de Articulação e Gestão de Políticas Públicas para Promoção da Democracia
Participativa}

\vspace{1.5cm}

{\bf \large Produto \ProductNumber\ -\ \ProductDescription}

\vspace{1.5cm}

\ProductSubtitle

\vspace{2.5cm}

\MyName

\vspace{1.5cm}

}

\includegraphics[scale=0.04]{brasao.png} \\
{\bf \small \NomeSecretaria}

\end{center}
\restoregeometry
\newpage

\input{folhaobjetocontratacao.tex}
\newgeometry{hmargin=3cm,vmargin=1.5cm}
\addtolength{\topmargin}{2.5cm}
\thispagestyle{empty}


\makebox{

\MySurnameForename \\
	\ProductTitle: \ProductSubtitle\ / \ContractYear.
\newline
	Total de folhas: \pageref{LastPage}

	Supervisor:
	Secretaria (Indicar SE/SNAS/SNARPS/SNJ etc.)
Secretaria-Geral da Presidência da República
	Palavras-chave:
}


{\raggedright \includegraphics{licenca-cc-by-nc.png} \ Esta obra é licenciada sob
uma licença Creative Commons - Atribuição-NãoComercial. 4.0 Internacional.}

\restoregeometry
\newpage

\tableofcontents
\newpage

\begin{abstract}
... \\

{\bf Palavras-chave:} \PalavrasChave.
\end{abstract}
\newpage

\section{Introdução}

Em consonância com os objetivos e cronograma previsto no âmbito do
projeto BRA/12/018:
\textbf{Desenvolvimento de Metodologias de Articulação e Gestão de
Políticas Públicas para Promoção da Democracia Participativa},
firmado entre a Secretaria-Geral da Presidência da República
(SG/PR) e o Programa das Nações Unidas para o Desenvolvimento (PNUD),
o presente documento apresenta \ProductDescription.

Essa proposta está configurada como produto \ProductNumber~da consultoria técnica
para especificação da construção dos códigos das metodologias de
organização da informação e interação participativa do portal de
participação social.

\section{Desenvolvimento}

O Participa.br é a Plataforma Federal da Participação Social. Trata-se de mais
um espaço para participação social no Brasil, escuta e diálogo entre o Governo
Federal e a Sociedade Civil. 

A plataforma, totalmente desenvolvida em software livre, tem como missão
desenvolver práticas inovadoras de participação via internet e oferta de
espaços de manifestação e debate para qualquer cidadão ou organização, com o
intuito de construir políticas públicas cada vez mais eficazes e efetivas.

O Participa.br é desenvolvido sob a plataforma para redes sociais Noosfero.

\subsection{O Noosfero}

O Noosfero\cite{noosfero} é uma plataforma web livre para redes sociais e de
economia solidária que possui as funcionalidades de Blog, e-Portfolios, CMS,
RSS, discussão temática, agenda de eventos e inteligência econômica
colaborativa num mesmo sistema! O Noosfero utiliza a linguagem de programação
Ruby com framework Rails e, portanto, suporta bancos de dados, PostgreSQL,
MySQL, SQLite entre outros.

Noosfero é um Software Livre e licenciado sob a GNU Affero General Public
License (AGPL), versão 3.

\subsection{Diaspora}

\section{Conclusão}

Neste documento foi apresentado um \ProductDescription

Lembramos que para tornar o Portal de Consulta Pública realmente um canal de
consulta e participação popular na discussão e na definição da agenda
prioritária do país, é necessário que além de documentação faça-se um esforço
de movimentar as pessoar fora do ambiente virtual, para que haja um
engajamento no uso e contribuição deste projeto de forma consistente e perene.

\newpage
\bibliography{bibliografia}
\newpage
\listoffigures
\newpage
\printindex
\newpage
\definecolor{lightgrey}{rgb}{0.95,0.95,0.95}
\lstset{language=Ruby,basicstyle=\small\ttfamily,backgroundcolor=\color{lightgrey}}

\section{Anexos}

\subsection{Exemplo de código do plugin Noosfero para pairwise}

\lstinputlisting{pairwise_content.rb}

\subsection{Exemplo de código XML retornado pelo pairwise-api}

\begin{lstlisting}
<?xml version="1.0" encoding="UTF-8"?>
<prompt>
  <created-at type="datetime">2010-07-01T23:48:01+00:00</created-at>
  <id type="integer">1</id>
  <left-choice-id type="integer">10</left-choice-id>
  <question-id type="integer">7</question-id>
  <right-choice-id type="integer">9</right-choice-id>
  <tracking nil="true"></tracking>
  <updated-at type="datetime">2010-07-01T23:48:01+00:00</updated-at>
  <votes-count type="integer">0</votes-count>
  <left-choice-text>bar</left-choice-text>
  <right-choice-text>foo</right-choice-text>
</prompt>
\end{lstlisting}

\appendix
\appendixpage
\section{Foo bar}
\label{foobar}

%\lstinputlisting{observatorio.rb}


\end{document}
