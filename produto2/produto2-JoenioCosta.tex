\documentclass[11pt]{article}
\renewcommand{\rmdefault}{ptm}
\usepackage[scaled=0.92]{helvet}
\usepackage{courier,xcolor,colortbl,listings,parskip,graphicx,fancyvrb,fancyhdr,lastpage}
\usepackage{float,framed}
\normalfont
\usepackage[T1]{fontenc}
%\setlength{\parskip}{7pt}
\usepackage[toc,page]{appendix}
\usepackage[hmargin=2.5cm,vmargin=2cm]{geometry}
\usepackage[utf8]{inputenc}
\usepackage[brazil]{babel}
\usepackage{fancyvrb}
\pagestyle{fancy}
\setlength{\headheight}{120pt}
\setlength{\headsep}{30pt}
\setlength{\textheight}{550pt}
\renewcommand{\headrulewidth}{0pt}
\lhead{}
\rhead{}
\chead{\includegraphics{brasao.jpg}\\
        \large \textbf{PRESIDÊNCIA DA REPÚBLICA}\\
        \large SECRETARIA-GERAL\\
        \large Secretaria-Executiva}
\cfoot{}
\rfoot{\thepage /\pageref{LastPage}}
\hyphenation{par-ti-ci-pa-ção}
\bibliographystyle{ieeetr}

\newcommand{\MyName}{Joenio Marques da Costa}
\newcommand{\MyEmail}{joenio@colivre.coop.br}
\newcommand{\ContractNumber}{2013/000564}
\newcommand{\ProjectCode}{Projeto PNUD BRA/12/018}
\newcommand{\NomeSecretaria}{Secretaria Geral da Presidência da República}
\newcommand{\SiglaSecretaria}{SG/PR}
\newcommand{\ProductNumber}{02}
\newcommand{\ProductDescription}{"Documento com análise de arquiteturas de
        sistemas de identidade distribuída, estratégia de implantação
        considerando os sites parceiros e contendo propostas de códigos."
}
\newcommand{\MesEntrega}{Abril de 2014}
\newcommand{\DiaEntrega}{21}

\begin{document}
\lstset{language=Ruby}
\definecolor{light-gray}{gray}{0.95}
\lstdefinestyle{codeFrame}{backgroundcolor=\color{light-gray},frame=lines}

\input{aprovacao.tex}
\clearpage
\input{identificacao.tex}
\clearpage

\tableofcontents
\clearpage
\listoffigures

\clearpage

\section{Apresentação}

Em consonância com os objetivos e cronograma previsto no âmbito do
projeto BRA/12/018:
\textbf{Desenvolvimento de Metodologias de Articulação e Gestão de
Políticas Públicas para Promoção da Democracia Participativa},
firmado entre a Secretaria-Geral da Presidência da República
(SG/PR) e o Programa das Nações Unidas para o Desenvolvimento (PNUD),
o presente documento apresenta um guia de codificação "coding guidelines" para
o desenvolvimento do código do portal objetivando o reaproveitamento de código
e o fomento à formação de comunidades em torno dos módulos, bem como tutoriais
para implementação local das soluções.

Essa proposta está configurada como produto \ProductNumber~da consultoria técnica
para especificação da construção dos códigos das metodologias de
organização da informação e interação participativa do portal de
participação social.


\section{SSO - Single Sign-on}

\subsection{O que é SSO?}

Single Sign-on, ou Web Browser Single Sign-on é a propriedade de controle de
acesso a sistemas de software relacionados, mas independentes. Com esta
propriedade o usuário loga apenas uma vez e ganha acesso a outros sistemas que
façam parte do mesmo ambiente de SSO sem necessidade de fornecer usuário/senha
diversas vezes. Reciprocamente, single sign-off é a propriedade do usuário
deslogar e automaticamente deslogar de outros sistemas que façam parte deste
mesmo ambiente.

Usualmente, ambientes SSO compartilham servidores de autenticação para servir
cada aplicação, com objetivo de autenticar e garantir que usuários não
necessitem entrar com usuário/senha mais de uma
vez~\footnote{http://en.wikipedia.org/wiki/Single\_sign-on}. Estes servidores
fornecem serviços de autenticação em rede para aplicações externas, a
autenticação pode ser feita por diversos métodos, mas normalmente usa-se
usuário/senha~\footnote{http://en.wikipedia.org/wiki/Authentication\_server} em
aplicações Web.

O uso de SSO aumenta drasticamente o impacto negativo em caso de roubo de
usuário/senha, uma vez que o acesso a esta informação possibilita acesso a
diversos sistemas, portanto a proteção dessas informações devem ser redobradas.
É preciso também ter cuidado com a disponibilidade deste serviço, uma vez
que sua queda implica em indisponibilidade dos serviços que fazem parte do
ambiente de SSO.

\subsection{SSO no Noosfero}

O Noosfero não implementa mecanismos de SSO, nem há referências na comunidade
de utilização dele num ambiente de SSO. O mecanismo de autenticação presente
no Noosfero está implementado nos seguintes arquivos:

\begin{itemize}
  \item noosfero/lib/authenticated\_system.rb
  \item noosfero/app/controllers/public/account\_controller.rb
  \item noosfero/app/model/user.rb
\end{itemize}

Esta implementação presente no {\it core} do Noosfero realiza autenticação
através de usuário/senha, e armazenado estas informações em banco de dados de
forma encriptada. É possível alterar o método de autenticação através de
plugins, um exemplo é o plugin Ldap distribuído junto ao Noosfero em:

\begin{itemize}
  \item noosfero/plugins/ldap
\end{itemize}

Este plugin possibilita realizar autenticação a partir de um servidor LDAP.

\subsection{Qual problema SSO resolve?}

SSO resolve um problema bem comum e conhecido, o usuário de um serviço Web
(site, sistema, rede social, etc) quer logar apenas uma vez e manter sua
sessão entre diversos serviços (outros sites, outras redes, etc) sem
necessidade de fornecer seus dados de acesso uma segunda vez.

Este problema é resultado de uma política segurança e privacidade implementada
nos Navegadores Web, esta política, chamada Same Origin
Policy~\footnote{https://en.wikipedia.org/wiki/Same-origin\_policy} é uma
recomendação do W3C e previne que documentos em diferentes domínios afetem e
compartilhem dados com outro domínio, isso previne, por exemplo, ataques de
cross-site scripting.

Inúmeras soluções foram propostas para contornar esta limitação, JSONP, CORS,
easyXDM, entre
outras~\footnote{http://stackoverflow.com/questions/7094967/single-sign-on-with-ajax-in-same-origin-policy-world-effective-solutions}, todas elas se tornaram obsoletas após a recomendação do
W3C chamada "Web Messaging"~\footnote{http://www.w3.org/TR/webmessaging/}, uma
técnica que permite documentos em diferentes domínios compartilar dados. A
maioria dos Navegadores Web atuais implementam Web
Messaging~\footnote{http://en.wikipedia.org/wiki/Web\_Messaging}.

% Ver mais em:
% http://security.stackexchange.com/questions/36753/designing-single-sign-on-with-jsonp-cors
% Exemplo simples e prático sobre funcionamento do Web Messaging:
% http://openblog.github.io/2013/02/25/html5-web-messaging-api/

\subsection{Como funciona?}

Existem várias tecnologias para implementar SSO: Kerberos, Smart card, OTP,
entre outras. Dependendo da solução as seguintes estratégias podem ser
utilizadas:

\begin{itemize}
  \item{As credenciais do usuário são passadas para um segundo domínio
        quando este acessa o segundo domínio.}
  \item{As credencias do usuário são obtidas de uma base de informação de
        gerenciamento de single sign-on.}
  \item{São estabelecidas várias sessões de login com todas as aplicações no
        momento em que se estabelece a primeira sessão.}
  \item{Armazenar temporariamente no cache e usar ao fazer requisição num
        segundo domínio.}
\end{itemize}

(daqui para baixo são informações anotadas de forma desordenada)

Fonte: http://www.opengroup.org/security/sso/sso\_intro.htm

Exemplo de implementação de SSO para múltiplos domínios da solução SiteMinder (http://www.ca.com/br/products/detail/ca-siteminder.aspx),
solução da "ca technologies" para implantar SSO, federação, etc.

* Usuário autentica uma vez.
* O navegador faz cache da autentitação e seta um cookie com informações de single sign-on
* O cookie fornece informações de sessão, assim o usuário pode acessar outros sites sem re-autenticar

Imagem com diagrama dessa solução:
http://support.ca.com/cadocs/0/CA\%20SiteMinder\%20r12\%205-ENU/Bookshelf\_Files/HTML/idocs/256303.png

Fonte:
http://support.ca.com/cadocs/0/CA\%20SiteMinder\%20r12\%205-ENU/Bookshelf\_Files/HTML/idocs/256655.html

\subsection{Quais soluções existem?}

Lista de implementações para SSO:
   * http://en.wikipedia.org/wiki/List\_of\_single\_sign-on\_implementations

Abaixo apenas as soluções livres.

Accounts \& SSO - Solução de SSO para computadores (apenas cliente)
http://en.wikipedia.org/wiki/Accounts\_\%26\_SSO

Central Authentication Service (CAS) - Protocolo Web (servidor) de SSO
http://en.wikipedia.org/wiki/Central\_Authentication\_Service

Como funciona:
http://rubycas.github.io/images/basic\_cas\_single\_signon\_mechanism\_diagram.png

(tem pacote Debian)

Distributed Access Control System (DACS) - SSO leve e controle de acesso basead em papéis para WEB
http://en.wikipedia.org/wiki/Distributed\_Access\_Control\_System\_(DACS)

(não inclui sistema de autenticação mas suporta muitos métodos, X.509, PAM, LDAP, etc...)


Enterprise Sign On Engine - Plataforma de SSO, controle de acesso e federação
http://en.wikipedia.org/wiki/Enterprise\_Sign\_On\_Engine

FreeIPA - Solução da RedHat para SSO, Policy and Audit, ...
http://en.wikipedia.org/wiki/FreeIPA

IBM Enterprise Identity Mapping - 
http://en.wikipedia.org/wiki/IBM\_Enterprise\_Identity\_Mapping

JBoss SSO - single sign-on, sign-off, federação
http://en.wikipedia.org/wiki/JBoss\_SSO

JOSSO - SSO para aplicações web, Java EE
http://en.wikipedia.org/wiki/JOSSO

Kerberos - Protocolo de autenticação em rede
http://en.wikipedia.org/wiki/Kerberos\_(protocol)

OpenAM - controle de acesso e federação, engloba uma série de soluções
http://en.wikipedia.org/wiki/OpenAM

Mozilla Persona - Não é SSO, mas sim sistema de autenticação descentralizado
http://en.wikipedia.org/wiki/Mozilla\_Persona

Pubcookie - Protocolo (e software) de SSO
http://en.wikipedia.org/wiki/Pubcookie

SAML - Linguagem de marcação para definir comunicação sobre autenticação e autorização
http://en.wikipedia.org/wiki/Security\_Assertion\_Markup\_Language

Shibboleth - SSO e autenticação
http://en.wikipedia.org/wiki/Shibboleth\_(Internet2)

ZXID - kit de gerenciamento de identidade SAML 2.0
http://en.wikipedia.org/wiki/ZXID


* Ler sobre protocolo CAS: http://www.jasig.org/cas/protocol


Existe um protocolo para implementar SSO:
   * http://en.wikipedia.org/wiki/Central\_Authentication\_Service

Thread sobre SSO com Persona:
   * https://groups.google.com/forum/\#!topic/mozilla.dev.identity/oNseXZxbVUQ


\subsection{Discussão sobre vantagens desvantagens de cada solução}

Riscos:
* https://blog.perfectcloud.io/does-facebook-federation-have-your-best-interests-at-heart-2/

Falha grave de segurança:
* http://research.microsoft.com/apps/pubs/default.aspx?id=160659

2. Identidade Digital / Autenticação

Uma solução de SSO não implica em serviço de identidade centralizada, ou seja, os sites envolvidos em SSO não
necessariamente utilizam um sistema de autenticação em de um servidor de identidade contralizada


O que é?
Resolve qual problema?
Como funciona?
Quais soluções existem?

OAuth
OpenID
OpenID Connect
Etc...

Comparativo entre soluções (vantagens desvantagens)

3. Iniciativas em andamento (Governo e Comunidade)

4. Proposta de solução para o Participa.BR (qual caminho tomar)

4.1 Objetivo

Nosso objetivo é que esse arranjo de confiança do Participa.br possa ser utilizado por plataformas de participação social do governo (gov) e da sociedade (org). Assim que implementado conectaríamos:

* Participatório (gov)
* Cidade Democrática (org)
* Cultura Educa (cc/org)

O caminho de implementação pode ser SLTI (gov) ou Serpro (gov/com).


\section{Considerações finais}

Neste documento foi apresentado um \ProductDescription

Lembramos que para tornar o Portal de Consulta Pública realmente um canal de
consulta e participação popular na discussão e na definição da agenda
prioritária do país, é necessário que além de documentação faça-se um esforço
de movimentar as pessoar fora do ambiente virtual, para que haja um
engajamento no uso e contribuição deste projeto de forma consistente e perene.

\vspace{1cm}

Sem mais nada a acrescentar, coloco-me à disposição.

\vspace{1cm}

\begin{minipage}{\textwidth}
  Brasília/DF, \DiaEntrega \ de \MesEntrega\\[1cm]
  \textbf{\MyName}\\
  \small Consultor do PNUD
\end{minipage}

\end{document}
