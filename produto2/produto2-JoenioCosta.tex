\documentclass[11pt]{article}
\renewcommand{\rmdefault}{ptm}
\usepackage[scaled=0.92]{helvet}
\usepackage{courier,xcolor,colortbl,listings,parskip,graphicx,fancyvrb,fancyhdr,lastpage}
\usepackage{float,framed}
\normalfont
\usepackage[T1]{fontenc}
%\setlength{\parskip}{7pt}
\usepackage[toc,page]{appendix}
\usepackage[hmargin=2.5cm,vmargin=2cm]{geometry}
\usepackage[utf8]{inputenc}
\usepackage[brazil]{babel}
\usepackage{fancyvrb}
\pagestyle{fancy}
\setlength{\headheight}{120pt}
\setlength{\headsep}{30pt}
\setlength{\textheight}{550pt}
\renewcommand{\headrulewidth}{0pt}
\lhead{}
\rhead{}
\chead{\includegraphics{brasao.jpg}\\
        \large \textbf{PRESIDÊNCIA DA REPÚBLICA}\\
        \large SECRETARIA-GERAL\\
        \large Secretaria-Executiva}
\cfoot{}
\rfoot{\thepage /\pageref{LastPage}}
\hyphenation{par-ti-ci-pa-ção}
\bibliographystyle{ieeetr}

\newcommand{\MyName}{Joenio Marques da Costa}
\newcommand{\MyEmail}{joenio@colivre.coop.br}
\newcommand{\ContractNumber}{2013/000564}
\newcommand{\ProjectCode}{Projeto PNUD BRA/12/018}
\newcommand{\NomeSecretaria}{Secretaria Geral da Presidência da República}
\newcommand{\SiglaSecretaria}{SG/PR}
\newcommand{\ProductNumber}{02}
\newcommand{\ProductDescription}{"Documento com análise de arquiteturas de
        sistemas de identidade distribuída, estratégia de implantação
        considerando os sites parceiros e contendo propostas de códigos."
}
\newcommand{\MesEntrega}{Abril de 2014}
\newcommand{\DiaEntrega}{21}

\begin{document}
\lstset{language=Ruby}
\definecolor{light-gray}{gray}{0.95}
\lstdefinestyle{codeFrame}{backgroundcolor=\color{light-gray},frame=lines}

\input{aprovacao.tex}
\clearpage
\input{identificacao.tex}
\clearpage

\tableofcontents
\clearpage
\listoffigures

\clearpage

\section{Apresentação}

Em consonância com os objetivos e cronograma previsto no âmbito do
projeto BRA/12/018:
\textbf{Desenvolvimento de Metodologias de Articulação e Gestão de
Políticas Públicas para Promoção da Democracia Participativa},
firmado entre a Secretaria-Geral da Presidência da República
(SG/PR) e o Programa das Nações Unidas para o Desenvolvimento (PNUD),
o presente documento apresenta um guia de codificação "coding guidelines" para
o desenvolvimento do código do portal objetivando o reaproveitamento de código
e o fomento à formação de comunidades em torno dos módulos, bem como tutoriais
para implementação local das soluções.

Essa proposta está configurada como produto \ProductNumber~da consultoria técnica
para especificação da construção dos códigos das metodologias de
organização da informação e interação participativa do portal de
participação social.


\section{SSO - Single Sign-on}

\subsection{O que é SSO?}

Single Sign-on, ou Web Browser Single Sign-on é a propriedade de controle de
acesso a sistemas Web onde usuários efetuam login apenas uma vez e ganham acesso a
outros sistemas que tenham relação com tal sistema sem a necessidade de fornecer
suas credenciais de autenticação uma segunda vez. Os sistemas que possuem
relação são definidos previamente pela implementação e configuração do
ambiente de SSO e é feito pelos desenvolvedores e administradores dos sistemas envolvidos.
Analogamente, single sign-off é a propriedade inversa, onde o usuário finaliza
sua sessão de login em um sistema e de forma automática ele é deslogado também
dos outros sistemas que façam parte deste mesmo ambiente.

Usualmente, ambientes SSO compartilham servidores de autenticação para servir
cada aplicação, com objetivo de autenticar e garantir que usuários não
necessitem entrar com suas credenciais de autenticação mais de uma
vez\footnote{http://en.wikipedia.org/wiki/Single\_sign-on}. Estes servidores
fornecem serviços de autenticação em rede para aplicações externas, a
autenticação pode ser feita por diversos métodos, mas normalmente usa-se
usuário/senha\footnote{http://en.wikipedia.org/wiki/Authentication\_server} em
aplicações Web.

O uso de SSO aumenta drasticamente o impacto negativo em caso de roubo de
informações, uma vez que o acesso a esta informação possibilita acesso a
diversos sistemas, portanto a proteção dessas informações devem ser redobradas.
É preciso também ter cuidado com a disponibilidade do serviço, uma vez
que sua queda implica em indisponibilidade dos serviços que fazem parte do
ambiente de SSO.

\subsection{SSO no Noosfero}

O Noosfero não implementa mecanismos de SSO, nem há referências na comunidade
de utilização dele num ambiente como este. O mecanismo de autenticação presente
no Noosfero está implementado nos seguintes arquivos:

\begin{itemize}
  \item noosfero/lib/authenticated\_system.rb
  \item noosfero/app/controllers/public/account\_controller.rb
  \item noosfero/app/model/user.rb
\end{itemize}

Esta implementação presente no {\it core} do Noosfero realiza autenticação
através de usuário/senha e armazena estas informações em banco de dados de
forma encriptada. É possível alterar o método de autenticação através de
plugins, um exemplo é o plugin Ldap distribuído junto ao Noosfero em:

\begin{itemize}
  \item noosfero/plugins/ldap
\end{itemize}

Este plugin possibilita realizar autenticação a partir de um servidor LDAP.

\subsection{Qual problema SSO resolve?}

SSO resolve um problema bem comum e conhecido, o usuário de um serviço Web
(site, sistema, rede social, etc) quer logar apenas uma vez e manter sua
sessão entre diversos serviços (outros sites, outras redes, etc) sem
necessidade de fornecer seus dados de acesso uma segunda vez.

A solução para esta questão precisa lidar com uma política segurança e
privacidade implementada nos Navegadores Web, esta política, chamada Same
Origin Policy\footnote{https://en.wikipedia.org/wiki/Same-origin\_policy} é
uma recomendação do W3C e previne que documentos em diferentes domínios afetem
e compartilhem dados com outros domínios, isso previne, por exemplo, ataques de
cross-site scripting.

Inúmeras soluções foram propostas para contornar esta política, JSONP, CORS,
easyXDM, entre
outras\footnote{http://stackoverflow.com/questions/7094967/single-sign-on-with-ajax-in-same-origin-policy-world-effective-solutions}, todas elas se tornaram obsoletas após a recomendação do
W3C chamada "Web Messaging"\footnote{http://www.w3.org/TR/webmessaging/}, uma
técnica que permite documentos em diferentes domínios compartilhar dados. A
maioria dos Navegadores Web atuais implementam Web
Messaging\footnote{http://en.wikipedia.org/wiki/Web\_Messaging}.

% Ver mais em:
% http://security.stackexchange.com/questions/36753/designing-single-sign-on-with-jsonp-cors
% Exemplo simples e prático sobre funcionamento do Web Messaging:
% http://openblog.github.io/2013/02/25/html5-web-messaging-api/

\subsection{Como SSO funciona?}

Existem muitas formas de implementar SSO: Kerberos, Smart card, CAS, OTP,
entre outras, cada uma com sua própria
estratégia\footnote{http://www.opengroup.org/security/sso/sso\_intro.htm}. A solução
SiteMinder\footnote{http://www.ca.com/br/products/detail/ca-siteminder.aspx}
da "CA Technologies", por exemplo, implementa SSO da seguinte
forma\footnote{http://support.ca.com/cadocs/0/CA\%20SiteMinder\%20r12\%205-ENU/Bookshelf\_Files/HTML/idocs/256655.html}:

\begin{itemize}
  \item{Usuário autentica uma vez}
  \item{O navegador faz cache da autenticação e seta um cookie com
        informações de single sign-on}
  \item{O cookie fornece informações de sessão, assim o usuário pode acessar
        outros sites sem necessidade de re-autenticar}
\end{itemize}

A Figura~\ref{fig:sso-siteminder} traz um diagrama exemplificando esta
solução. A solução SiteMinder é um sistema centralizado de gerenciamento de
acesso Web da empresa "CA Technologies", que implementa uma série de soluções,
além de SSO.

\begin{figure}[h]
\center
\includegraphics[scale=0.6]{sso-siteminder.png}
\caption{Exemplo de implementação de SSO do SiteMinder}
\label{fig:sso-siteminder}
\end{figure}

\subsection{Quais soluções de SSO existem?}

A seguir é apresentada uma lista de soluções em software livre para SSO
elaborada com base no artigo da Wikipedia em:

\begin{itemize}
  \item http://en.wikipedia.org/wiki/List\_of\_single\_sign-on\_implementations
\end{itemize}

\subsubsection{Accounts \& SSO}

Framework contendo um conjunto de componentes e bibliotecas para autenticação
de contas de usuários online para clientes Desktops, sistemas Linux e POSIX.

Mais em:
\begin{itemize}
  \item{http://en.wikipedia.org/wiki/Accounts\_\&\_SSO}
\end{itemize}

{\bf Avaliação:} {\it Não é uma opção válida para ser implantado no portal de
participação social Participa.br pois é voltado para clientes desktop.}

\subsubsection{Central Authentication Service (CAS)}

Protocolo de Single Sign-on para a Web. O nome CAS refere-se também a uma
implementação deste mesmo protocolo. O fluxo durante a autenticação é o
seguinte:

\begin{itemize}
  \item{Cliente visita uma aplicação que requisita autenticação}
  \item{A aplicação redireciona o Cliente para o CAS}
  \item{CAS valida atenticidade do Cliente (geralmente contra um banco, Kerberos, Active Directory, etc)}
  \item{Se a autenticação tem sucesso, o CAS retorna o cliente para a aplicação, passando um ticket de segurança}
  \item{A aplicação valida o ticket contactando o CAS}
  \item{O CAS dá a informação com segurança à aplicação que o usuário foi autenticado com sucesso}
\end{itemize}

A implementação oficial do CAS é em Java e é mantido, hoje, pelo grupo
JASIG\footnote{http://www.jasig.org}, existem implementações oficiais de
cliente em várias linguagens, como: .NET, PHP, Perl, Apache, etc.

\begin{figure}[h]
\center
\includegraphics[scale=0.4]{sso-rubycas.png}
\caption{Diagrama do RubyCAS, implementação do protocolo CAS em Ruby}
\label{fig:sso-rubycas}
\end{figure}

O artigo {\it Approaches and challenges for a single sign-on enabled extranet
using Jasig CAS} descreve a experiência em configurar single sign-on em um
ambiente de intranet usando o CAS e outros software livres, e faz uma boa
avaliação das tecnologias envolvidas, como OpenID, OAuth, etc.

\begin{itemize}
  \item{http://openidentity.eu/2013/media/09\_Holzschuher\_Peinl\_2013\_SSO\_extranet\_CAS\_LNI.pdf}
\end{itemize}

%http://rubycas.github.io
%(tem pacote Debian)

Mais em:
\begin{itemize}
  \item{http://en.wikipedia.org/wiki/Central\_Authentication\_Service}
  \item{http://www.jasig.org/cas/protocol}
\end{itemize}

{\bf Avaliação:} {\it Implementação madura contendo um bom conjunto de
técnicas para implementação de SSO, incluindo suporte a OAuth, OpenID, SAML, LDAP,
etc. É uma opção recomendada.}

\subsubsection{Distributed Access Control System (DACS)}

DACS é um sistema de SSO leve combinado com mecanismos de autenticação e
controle de acesso para Web escrito em C/C++. Possui suporte para integrar com
diversos mecanismos de autenticação, como X.509, PAM, LDAP, etc. Possui módulo
de autenticação para servidor Web Apache.

%http://pdf.aminer.org/000/309/229/a_single_sign_on_protocol_for_distributed_web_applications_based.pdf

O Debian utiliza DACS para prover SSO entre alguns dos seus servidores de
desenvolvimento do projeto.

Mais em:
\begin{itemize}
  \item{http://en.wikipedia.org/wiki/Distributed\_Access\_Control\_System\_(DACS)}
  \item{https://wiki.debian.org/DebianSingleSignOn}
\end{itemize}

{\bf Avaliação:} {\it ?????.}

\subsubsection{Enterprise Sign On Engine}

Plataforma de SSO, controle de acesso e federação compatível com SAML 2.0
e parcialmente compatível com
XACML\footnote{http://en.wikipedia.org/wiki/XACML}.

Desenvolvido em Java, possui suporte a Tomcat, Apache e IIS.

Mais em:
\begin{itemize}
  \item{http://en.wikipedia.org/wiki/Enterprise\_Sign\_On\_Engine}
\end{itemize}

{\bf Avaliação:} {\it Nenhuma referência encontrada sobre seu uso em produção,
não recomendado para o portal de participação social Participa.br.}

\subsubsection{FreeIPA}

Solução da RedHat para SSO e "Policy and Audit". É comparável a solução
"Novell's Identity Manager" ou "Microsoft's Active Directory" pois tem
objetivos e mecanismos similares.

Usa as soluções 389 Directory Server, MIT Kerberos 5, Apache HTTP e Python. A
partir da versão 3.0.0 adicionou suporte a Samba para integração com Microsoft
Active Directory.

Mais em:
\begin{itemize}
  \item{http://en.wikipedia.org/wiki/FreeIPA}
\end{itemize}

{\bf Avaliação:} {\it Esta solução é incialmente voltada para ambientes de
rede desktop e não atende aos requisitos do Participa.br.}

\subsubsection{IBM Enterprise Identity Mapping}

Framework para mapear identidades de usuários em várias plataformas distintas,
pouca informação disponível na Web.

Mais em:
\begin{itemize}
  \item{http://en.wikipedia.org/wiki/IBM\_Enterprise\_Identity\_Mapping}
\end{itemize}

{\bf Avaliação:} {\it Voltado apenas para integrar soluçoes da própria
IBM.}

\subsubsection{JBoss SSO}

Faz parte da suíte de soluções JBoss SOA, permite single sign-on, sign-off e
acesso federado a múltiplas aplicações e recursos computacionais em rede.

Dentre várias funcionalidades o JBoss SSO inclue:

\begin{itemize}
  \item{Integração entre aplicações e módulos baseados no padrão SAML}
  \item{Abordagem descentralizada}
  \item{Habilidade de conectar a diferentes sistemas de armazenamento}
\end{itemize}

Mais em:
\begin{itemize}
  \item{http://en.wikipedia.org/wiki/JBoss\_SSO}
\end{itemize}

{\bf Avaliação:} {\it Solução madura, possível alternativa a ser utilizada.}

\subsubsection{JOSSO}

Java Open Single Sign On (JOSSO) é uma solução de SSO para aplicações Web.
Baseado em Java EE, permite mútiplos servidores web autenticar
usuários através de suas credenciais. JOSSO se comunica com o armazenamento das
credenciais por LDAP ou JDBC e fornece interface via SOAP sob o protocolo HTTP
para permitir fácil integração com aplicações não-Java.

Mais em:
\begin{itemize}
  \item{http://en.wikipedia.org/wiki/JOSSO}
\end{itemize}

{\bf Avaliação:} {\it Solução madura, possível alternativa a ser utilizada.}

\subsubsection{Kerberos}

Protocolo de autenticação em rede baseado em 'tickets', permite comunicação
entre nós sob uma rede não-segura de forma segura. Foi projetado
principalmente com um modelo cliente-servidor, isso provê autenticação tanto
de usuários quanto de servidores. Veja na Figura~\ref{fig:kerberos} um exemplo
de negociação com o Kerberos.

\begin{figure}[h]
\center
\includegraphics[scale=0.6]{kerberos.png}
\caption{Diagrama de negociação do Kerberos}
\label{fig:kerberos}
\end{figure}

Mais em:
\begin{itemize}
  \item{http://en.wikipedia.org/wiki/Kerberos\_(protocol)}
\end{itemize}

{\bf Avaliação:} {\it Protocolo de autenticação, não é uma solução para SSO.}

\subsubsection{OpenAM}

Provê single sign-on de forma transparente em infraestrutura de redes. Escrito
em Java, suporte a mais de 20 tipos de autenticação, possui suporte a SAML e
implementa sistema de autorização baseado em XACML. Veja um exemplo na
Figura~\ref{fig:opensso} de uso do OpenAM em um portal de viagens.

\begin{figure}[h]
\center
\includegraphics[scale=0.5]{opensso.png}
\caption{Exemplo de federação com OpenAM para um portal de viagens}
\label{fig:opensso}
\end{figure}

Mais em:
\begin{itemize}
  \item{http://en.wikipedia.org/wiki/OpenAM}
\end{itemize}

{\bf Avaliação:} {\it Possível alternativa de ser utilizada, madura, bem
documentada e bastante utilizada, suporta: OAuth, SAML, Kerberos, LDAP, etc.}

\subsubsection{Pubcookie}

Protocolo (e software) de SSO, o processo de autenticação se dá da seguinte
forma:

\begin{itemize}
  \item{Quando usuário acessa a aplicação, Pubcookie seta 2 cookies, pré-sessão e concessão de requisição}
  \item{Redireciona usuário para página de login}
  \item{Usuário fornece login e senha, se o login for com sucesso, seta 2 cookies, login e concessão}
\end{itemize}

Mais em:
\begin{itemize}
  \item{http://en.wikipedia.org/wiki/Pubcookie}
\end{itemize}

{\bf Avaliação:} {\it O último release do projeto é de 2010, não recomendado
como possível solução de SSO para o portal Participa.br}

\subsubsection{SAML}

Linguagem de marcação para definir comunicação sobre autenticação e autorização

Security Assertion Markup Language (SAML) é uma linguagem de marcação baseada
em XML para troca de dados de autenticação e autorização definido pelo OASIS
Security Services Technical Committee. O SAML é principalmente desenvolvido
para ser aplicado em Web Browser Single Sign-on.

A especificação SAML define 3 papéis:
\begin{itemize}
  \item{Principal (geralmente um usuário)}
  \item{Provedor de identidade (IdP)}
  \item{Provedor de serviço (SP)}
\end{itemize}

A interação entre eles está representada na Figura~\ref{fig:saml2}.

\begin{figure}[h]
\center
\includegraphics[scale=0.5]{saml2.png}
\caption{Single Sign-on com SAML2}
\label{fig:saml2}
\end{figure}

Mais em:
\begin{itemize}
  \item{http://en.wikipedia.org/wiki/Security\_Assertion\_Markup\_Language}
\end{itemize}

{\bf Avaliação:} {\it Não uma solução de SSO em sí, é suportado por várias
soluções, é altamente recomendado que a solução adotada no Participa.br
suporte este padrão.}

\subsubsection{Shibboleth}

Middleware para SSO e autenticação baseado em SAML.

Mais em:
\begin{itemize}
  \item{http://en.wikipedia.org/wiki/Shibboleth\_(Internet2)}
\end{itemize}

{\bf Avaliação:} {\it É uma possível alternativa a ser utilizada no
Participa.br, tem boas referências de uso na prática como por exemplo a
iniciativa das universidades do Reuino Unido chamada OpenAthens.}

\subsubsection{ZXID}

Kit de gerenciamento de identidade SAML 2.0. Compatícel co SAML 2.0, Liberty
ID-WSF 2.0 e XACML 2.0. Implementado em C com, possui poucas dependencias
externas, fornece bibliotecas para PHP, Perl e Java via SWIG.

Mais em:
\begin{itemize}
  \item{http://en.wikipedia.org/wiki/ZXID}
\end{itemize}

{\bf Avaliação:} {\it Não recomendado.}

\section{IdP - Identity Provider}

\subsection{O que é IdP?}

Identity Provider, ou Provedor de Identidade, é o serviço responsável por
gerenciar informações de identidade entre sistemas, usuários ou outros atores,
provendo através de um módulo interno ou externo serviço de autenticação e
autorização, de forma segura, afim de verificar a autenticidade.

Um provedor de identidade fornece uma alternativa para que vários serviços web
distintos autentiquem seus usuários através dele, de forma que um usuário pode
ter apenas um login/senha e autenticar em vários serviços com este mesmo
login/senha. Isto não implica em Single Sign-on, pois com um provedor de
identidade apenas o usuário ainda precisa passar por uma etapa de
autenticação, num ambiente de SSO isto fica transparente e o usuário ao logar
num sistema Web não precisa passar pela etapa de autenticação ao acessar um
segundo serviço.

Uma solução de SSO é composta usualmente de ao menos 3 componentes:

\begin{itemize}
  \item{Usuário/cliente, geralmente usa-se os termo {\it Principal}}
  \item{Provedor de identidade, {\it IdP - Identity Provider}}
  \item{Provedor de serviço, {\it SP - Service Provider}}
\end{itemize}

Neste cenário o usuário autentica apenas uma vez e um token de segurança é
passado entre os sistemas participantes do ambiente de SSO. Normalmente
suportam os tipos de token de segurança mais comuns, como: SAML, SPNEGO,
X.509.

Um sistema de provedor de identidade (IdP) é normalmente parte de um ambiente
de SSO, usualmente é um primeiro passo para se ter Single Sign-on.

Mais sobre Identity Provider e os diversos conceitos envolvidos podem ser
encontrados em:
\begin{itemize}
  \item{http://en.wikipedia.org/wiki/Identity\_provider}
  \item{http://www.empowerid.com/learningcenter/technologies/service-identity-providers}
\end{itemize}

\subsection{IdP no Noosfero}

O Noosfero não implementa oficialmente integração com nenhum protocolo ou
tecnologia de provedor de identidade, no entando existe uma implementação
não-oficial em uso na rede Cirandas.net\footnote{http://cirandas.net} de uso
do OAuth e Mozilla Persona, o código fonte desta implementação, ainda não
integrada ao repositório oficial do Noosfero, encontra-se em:

\begin{itemize}
  \item{https://github.com/CIRANDAS/noosfero-ecosol/tree/master/plugins/oauth}
\end{itemize}

\subsection{Qual problema IdP resolve?}

Um usuário cria apenas um registro login/senha e acessa múltiplos serviços
através deste mesmo login/senha, sem necessidade de ter várias senhas em cada
serviço Web que deseje acessar.

\subsection{Quais soluções de IdP existem?}

\subsubsection{Mozilla Persona}

O Mozilla Persona é um sistema de autenticação descentralizado, projeto
iniciado em 2011 e compartilha alguns dos objetivos de sistemas similares como
OpenID ou Facebook Connect, mas com algumas diferenças:

\begin{itemize}
  \item{Usa endereços de email como identificador}
  \item{Foco na privacidade do usuário}
  \item{Forte integração com navegadores web}
\end{itemize}

É baseado no protocolo BrowserID, proposto pela própria
Mozilla\footnote{http://identity.mozilla.com/post/7616727542/introducing-browserid-a-better-way-to-sign-in}.
A privacidade é um ponto central de preocupação neste protocolo, ele propõe que
nem o provedor de identidade nem o outros servidores saibam quais sites o
usuário está acessando.

Mais em:
\begin{itemize}
  \item{http://en.wikipedia.org/wiki/Mozilla\_Persona}
  \item{https://login.persona.org}
  \item{https://developer.mozilla.org/en-US/Persona/Implementing\_a\_Persona\_IdP}
\end{itemize}

{\bf Avaliação:} {\it Altamente recomendado.}

\subsubsection{OAuth}

OAuth é um padrão aberto para autorização, desenhado especialmente para
funcionar sob o protocolo HTTP, essencialmente permite acesso a tokens gerados
por servidores de autorização, o cliente usa tal token para acessar serviços
protegidos. Ele é complementar, apesar de distinto, ao OpenID.

A maioria dos grandes serviços para Web implementam OAuth, como: Google,
Facebook, Yahoo, AOL, Microsoft, PayPal, MySpace, Flickr, entre outros.

Mais em:
\begin{itemize}
  \item{http://en.wikipedia.org/wiki/OAuth}
\end{itemize}

Existem algumas preocupações com o OAuth, alguns dos autores iniciais, como o
Eran Hammer por exemplo, saíram do projeto
apontando\footnote{http://hueniverse.com/2012/07/26/oauth-2-0-and-the-road-to-hell}
uma série de falhas e apontam um novo caminho, como o
Oz\footnote{https://github.com/hueniverse/oz} e
Hawk\footnote{https://github.com/hueniverse/hawk} por exemplo, um protocolo de
autorização e um esquema de autenticação HTTP, respectivamente.

{\bf Avaliação:} {\it Apesar dos problemas apontados por alguns
desenvolvedores do projeto, OAuth é praticamente um padrão e seu uso é
altamente recomendado no portal de participação Participa.br.}

\subsubsection{OpenID}

Sistema de identificação baseado em URL, permite autenticação de usuários a
partir de sistemas de autenticação parceiros, usuários podem criar seu acesso
onde desejar e logar onde quer que OpenID seja suportado. Alguns exemplos de
provedores OpenID: Google, Yahoo!, PayPal, BBC, AOL, LiveJournal e MySpace.

No OpenID os usuários são identificados através de URLs, isto foi fortemente
influencido pelo sistema
LID\footnote{http://en.wikipedia.org/wiki/Light-Weight\_Identity}, um sistema
online de gerenciamento de identidade digital desenvolvido como parte do
NetMesh.

Mais em:
\begin{itemize}
  \item{http://en.wikipedia.org/wiki/OpenID}
\end{itemize}

{\bf Avaliação:} {\it Recomendado.}

\subsubsection{OpenID Connect}

Camada de autenticação em cima do OAuth 2.0 promovido pelo OpenID Foundation.

Mais em:
\begin{itemize}
  \item{http://en.wikipedia.org/wiki/OpenID\_Connect}
  \item{http://openid.net/connect}
\end{itemize}

{\bf Avaliação:} {\it Recomendado.}

\section{Outras iniciativas}

\subsection{OpenAthens - Reino Unido}

Iniciativa de implantação de SSO no Reino Unido, iniciou em instituições de
ensino universidades e então pelas instituições de saúde, adota SAML e
interfaces via Shibboleth. Em funcionamento desde 1996 com mais de 4.5 milhões
de contas de usuários, usado para prover acesso a mais de 300 serviços web
distintos, entre universidades e serviços públicos. Desenvolvido pela empresa
Eduserv, uma empresa sem fins lucrativos sediada em Bath-UK.

Mais em:
\begin{itemize}
  \item{http://www.eduserv.org.uk/services/OpenAthens}
  \item{http://en.wikipedia.org/wiki/Athens\_access\_and\_identity\_management}
  \item{http://everything2.com/index.pl?node\_id=1888399}
\end{itemize}

\subsection{Microsoft account}

O Microsoft account, anteriormente conhecido como Windows Live ID é um
provedor de identidade para os serviços da Microsoft, está sendo cidado aqui
apenas para exemplificar uma solução não-livre em produção, existem outras,
como o próprio Facebook Connect. O Microsoft account implementa OpenID e é
também um provedor de identidade OpenID.

Mais em:
\begin{itemize}
  \item{http://en.wikipedia.org/wiki/Microsoft\_account}
\end{itemize}

\subsection{Liberty Alliance}

Iniciativa entre organizações para promover padrões de federação, IGF,
serviços de identidade, etc... submeteu para o OASIS Group a especificação do
SAML 2.0, propôs uma séries de soluções como: OpenAz, ZXID, etc... As
iniciativas deste grupo hoje estão sendo mantidas pelo Kantara Initiative.

Mais em:
\begin{itemize}
  \item{http://en.wikipedia.org/wiki/Liberty\_Alliance}
  \item{https://en.wikipedia.org/wiki/Kantara\_Initiative}
\end{itemize}

\section{Discussão}

Discussao sobre CAS x OAuth:
* http://stackoverflow.com/questions/2033026/sso-with-cas-or-oauth/3181557\#3181557

CAS centraliza a autenticação, deve ser usado quando todas as aplicações
autenticam numa mesma base de credenciais usuário/senha de usuários.

OpenID decentraliza a autenticação, deve ser usado para aceitar login de
qualquer provedor OpenID, mas a aplicação pode restringir quais provedores
OpenID aceitar no entando.

Nem CAS nem OpenID lidam com autorização nativamente.

OAuth lida com autorização, autoriza por exemplo que um site X possa efetuar
autenticação a partir de um serviço de autenticação de terceiros Y. OAuth é
sobre permitir usuário controlar como seus recursos serão acessados por
terceiros.

CAS a partir da versão 3.5 suporta OAuth cliente e
servidor\footnote{https://wiki.jasig.org/display/CASUM/OAuth}.

CASE: OpenSSO foi utilizado pela CPqD para implantas SSO entre diversar aplicações
da empresa https://blogs.oracle.com/superpat/entry/opensso\_at\_cpqd

CAS é geralmente a escolhe preferida para grande organizações, onde se quer
centralizar a base de usuários, exemplo universidades.

Falha grave de segurança:
* http://research.microsoft.com/apps/pubs/default.aspx?id=160659

Discussao na visao do pessoal do DACS sobre o porque SSO é pouco adotado no
geral:
* http://dacs.dss.ca/about.html

\subsection{Iniciativas (Governo e Comunidade)}

\subsubsection{Login Cidadão}

Projeto piloto desenvolvido pela PROCERGS de um provedor de identidade com
base em OAuth para os serviços do governo do estado do Rio Grande do Sul.

Instancia do Login Cidadão rodandm em:
\begin{itemize}
  \item{https://meu.rs.gov.br}
\end{itemize}

Código fonte, projeto desenvolvido em PHP com framework symfony:
\begin{itemize}
  \item{https://github.com/PROCERGS/login-cidadao}
\end{itemize}

A equipe responsável pelo Login Cidadão esteve no Fisl15 apresentando a
palestra "Login Cidadão: Uma conta. Tudo o que o governo oferece", o vídeo
está disponível através do link abaixo:
\begin{itemize}
  \item{http://hemingway.softwarelivre.org/fisl15/high/41f/sala41f-high-201405081612.ogv}
\end{itemize}

\subsubsection{Id da Cultura}

O MinC (Ministério da Cultura) iniciou um projeto de provedor OpenID para
fornecer identidade centralizada para os serviços Web do MinC e parceiros, o
projeto é desenvolvido em Python e o código-fonte está disponível em:
\begin{itemize}
  \item{https://github.com/hacklabr/iddacultura-provider}
\end{itemize}

Esta implementação foi posteriormente aproveitada pelo projeto Mapas Culturais
do Estado de São Paulo e hoje está sendo utilizado neste projeto, o código
utilizado neste projeto está em:
\begin{itemize}
  \item{https://github.com/hacklabr/mapasculturais-openid}
\end{itemize}

\subsection{Proposta para o Participa.BR}

Qual caminho tomar?

Neste contexto de SSO, o Participa.BR tem a necessidade de criar um
arranjo de confiança entre alguns sites relacionados ao projeto, inicialmente
estes os sites que farão parte deste arranjo são, além do próprio Participa.br
são:

\begin{itemize}
  \item{Participatório (gov)}
  \item{Cidade Democrática (org)}
  \item{Cultura Educa (cc/org)}
\end{itemize}

Assim que implementado conectaríamos estes sites acima.

O caminho de implementação pode ser SLTI (gov) ou Serpro (gov/com).

Existem 2 caminhos distintos a adotar:
1) Implementar SSO logo de cara, usar OpenAM ou alguma outra solução de SSO; ou
2) Implementar apenas um provedor de identidade, com base no banco do
Participa.br, e permitir que os sites do arranjo loguem através disso

A opção 2) se mostra mais interessante, uma vez que ela é inicialmente menos
custosa de implementar, e parte do trabalho (senão todo o trabalho) é
aproveitado ao se implementar SSO de verdade futuramente.

Dentre as opções pesquisadas, Mozilla Persona se mostra bastante interessante,
uma vez que utiliza um identificador de usuário em formato de email
usuario@dominio.com, ao invés de uma URL como é feito no OpenID
http://dominio.com/usuario

O OpenAM como solução mostra por outro lado uma grande vantagem pois é uma
tecnologia já adotada pelo Serpro, o que facalitaria bastante sua implantação
caso a solução seja implementada por esta instituição.

OpenAM, provê uma solução completa, provedor de identidade, single sign-on, em
um só pacote, seria necessário apenas implementar no Noosfero uma forma de
autenticar através da base do OpenAM.

A alternativa mais apropriada é implementar inicialmente suporte a OAuth,
OpenID ou Mozilla Persona no Noosfero, implementar tanto o lado cliente quando
o lado servidor (provider), isto irá fornecer a infraestrutura básica para num
segundo momento implementar Single Sign-on de verdade.

O OpenAM tem suporte a OAuth, OpenID Connect, então é inteiramente possível
aproventar o trabalho feito no momento que for implantar SSO de verdade usando
OpenAM, caso seja esta a solução eleita.

\section{Considerações finais}

Neste documento foi apresentado um \ProductDescription

Lembramos que para tornar o Portal de Consulta Pública realmente um canal de
consulta e participação popular na discussão e na definição da agenda
prioritária do país, é necessário que além de documentação faça-se um esforço
de movimentar as pessoar fora do ambiente virtual, para que haja um
engajamento no uso e contribuição deste projeto de forma consistente e perene.

\vspace{1cm}

Sem mais nada a acrescentar, coloco-me à disposição.

\vspace{1cm}

\begin{minipage}{\textwidth}
  Brasília/DF, \DiaEntrega \ de \MesEntrega\\[1cm]
  \textbf{\MyName}\\
  \small Consultor do PNUD
\end{minipage}

\end{document}
