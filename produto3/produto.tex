\documentclass[12pt]{article}
\usepackage[usenames,dvipsnames]{color}
\usepackage{listings}
\usepackage{graphicx}
\usepackage[export]{adjustbox}
\usepackage{fancyhdr}
\usepackage{framed}
\usepackage[T1]{fontenc}
\usepackage[toc,page]{appendix}
\usepackage[utf8]{inputenc}
\usepackage[brazil]{babel}
\usepackage{fancyvrb}
\usepackage[hmargin=2cm,vmargin=2cm]{geometry}
\usepackage{lastpage}
\usepackage{makeidx}
\pagestyle{fancy}

% cabecalho e rodapé
\setlength{\headheight}{120pt}
\setlength{\textheight}{550pt}
\renewcommand{\headrulewidth}{0pt}
\lhead{\includegraphics[scale=0.03]{brasao.png}}
\rhead{\includegraphics[scale=0.4]{logo-pnud.png}}
\cfoot{\textbf{\ProjectCode\ - Inovando a democracia participativa}}
\rfoot{\thepage}

\hyphenation{par-ti-ci-pa-ção}
\bibliographystyle{ieeetr}

% definições sobre o autor e o produto
\newcommand{\MyName}{Joenio Marques da Costa}
\newcommand{\MySurnameForename}{Costa, Joenio}
\newcommand{\SupervisorName}{Gabriella Vieira Oliveira Gonçalves}
\newcommand{\MyEmail}{joenio@colivre.coop.br}
\newcommand{\ContractNumber}{2013/000564}
\newcommand{\ContractYear}{2013}
\newcommand{\ProjectCode}{Projeto BRA/12/018}
\newcommand{\NomeSecretaria}{Secretaria Geral da Presidência da República}
\newcommand{\SiglaSecretaria}{SG/PR}
\newcommand{\ProductNumber}{03}
\newcommand{\ProductTitle}{Título do Produto}
\newcommand{\ProductSubtitle}{Subtítulo do Produto}
\newcommand{\ProductDescription}{"Documento com proposta para desenvolvimento
  do código de 2 aplicativos para as trilhas de participação social
  priorizados nas reuniões do projeto contendo exemplos e códigos."
}
\newcommand{\ProductValue}{R\$ 18.000,00 (dezoito mil reais)}
\newcommand{\ObjetoContratacao}{"Construção dos códigos para comunidades e
  aplicativos do portal da participação social."
}
\newcommand{\DataEntrega}{18 de Julho de 2014}
\newcommand{\PalavrasChave}{palavra1, palavra2, palavra3, etc}

% lista de abreviações
\makeindex

\begin{document}

\newgeometry{hmargin=3cm,vmargin=1.5cm}
\begin{center}
\thispagestyle{empty}
{\color{MidnightBlue}

\includegraphics[scale=0.9]{logo-pnud.png}

\vspace{4cm}

{\bf \large \ProjectCode\ - Desenvolvimento de Metodologias
de Articulação e Gestão de Políticas Públicas para Promoção da Democracia
Participativa}

\vspace{1.5cm}

{\bf \large Produto \ProductNumber\ -\ \ProductDescription}

\vspace{1.5cm}

\ProductSubtitle

\vspace{2.5cm}

\MyName

\vspace{1.5cm}

}

\includegraphics[scale=0.04]{brasao.png} \\
{\bf \small \NomeSecretaria}

\end{center}
\restoregeometry
\newpage

\newgeometry{hmargin=3cm,vmargin=1.5cm}
\addtolength{\topmargin}{2.5cm}
\thispagestyle{empty}

{\bf \ProjectCode \ -} \ProductDescription

\vspace{2.5cm}

\begin{minipage}{\textwidth}
  {\bf Consultor(a): \MyName}
  \newline
  {\bf Contrato nº: \ContractNumber}
  \newline
  {\bf Produto / nº: \ProductNumber}
\end{minipage}

\vspace{2cm}

{\bf Assinatura do Consultor(a)}

\begin{framed}
\noindent Local e data: Brasília/DF, \line(1,0){20} \ de \line(1,0){100} \ de 2014
\newline
\newline
Assinatura do Consultor(a): \line(1,0){300}
\end{framed}

\vspace{1cm}

{\bf Assinatura do Supervisor(a)}

\begin{framed}
\noindent Atesto que os serviços foram prestados conforme estabelecido no
Contrato de Consultoria.
\newline
\newline
Local e data: Brasília/DF, \line(1,0){20} \ de \line(1,0){100} \ de 2014
\newline
\newline
Assinatura e Carimbo: \line(1,0){300}
\end{framed}

\vspace{1.5cm}

\begin{center}
\includegraphics[scale=0.04]{brasao.png} \\
{\bf \small \NomeSecretaria}
\end{center}

\restoregeometry
\newpage

\newgeometry{hmargin=3cm,vmargin=1.5cm}
\addtolength{\topmargin}{2.5cm}
\thispagestyle{empty}


\makebox{

\MySurnameForename \\
	\ProductTitle: \ProductSubtitle\ / \ContractYear.
\newline
	Total de folhas: \pageref{LastPage}

	Supervisor:
	Secretaria (Indicar SE/SNAS/SNARPS/SNJ etc.)
Secretaria-Geral da Presidência da República
	Palavras-chave:
}


{\raggedright \includegraphics{licenca-cc-by-nc.png} \ Esta obra é licenciada sob
uma licença Creative Commons - Atribuição-NãoComercial. 4.0 Internacional.}

\restoregeometry
\newpage

\tableofcontents
\newpage

\begin{abstract}
Resumo... \\

{\bf Palavras-chave:} \PalavrasChave.
\end{abstract}
\newpage

\section{Introdução}

Em consonância com os objetivos e cronograma previsto no âmbito do
projeto BRA/12/018:
\textbf{Desenvolvimento de Metodologias de Articulação e Gestão de
Políticas Públicas para Promoção da Democracia Participativa},
firmado entre a Secretaria-Geral da Presidência da República
(SG/PR) e o Programa das Nações Unidas para o Desenvolvimento (PNUD),
o presente documento apresenta \ProductDescription.

Essa proposta está configurada como produto \ProductNumber~da consultoria técnica
para especificação da construção dos códigos das metodologias de
organização da informação e interação participativa do portal de
participação social, Participa.br, desenvolvido utilizando a ferramenta
Noosfero.

\subsection{O Participa.br}

O Participa.br é a Plataforma Federal da Participação Social. Trata-se de mais
um espaço para participação social no Brasil, escuta e diálogo entre o Governo
Federal e a Sociedade Civil. 

A plataforma, totalmente desenvolvida em software livre, tem como missão
desenvolver práticas inovadoras de participação via internet e oferta de
espaços de manifestação e debate para qualquer cidadão ou organização, com o
intuito de construir políticas públicas cada vez mais eficazes e efetivas.

O Participa.br é desenvolvido sob a plataforma de software livre para redes
sociais Noosfero e conta com uma equipe multi-disciplinar entre jornalistas,
profissionais de design gráfico, desenvolvedores de software, pesquisadores e
gestores públicos.

\subsection{O Noosfero}

O Noosfero é uma plataforma web livre para redes sociais e de economia
solidária que possui as funcionalidades de Blog, e-Portfolios, CMS, RSS,
discussão temática, agenda de eventos e inteligência econômica colaborativa
em um mesmo sistema. O Noosfero utiliza a linguagem de programação Ruby com
framework Rails e suporta bancos de dados, PostgreSQL, MySQL, SQLite entre
outros.

Ele tem sido desenvolvido desde 2007 como um Software Livre e é licenciado sob
a GNU Affero General Public License (AGPL), versão 3.

\section{Desenvolvimento}

A plataforma de participação social, Participa.br, tem sido desenvolvida desde
de 2013 através de uma metodologia colaborativa e descentralizada, tendo como
ponto central de documentação e gestão de informações a plataforma
Gitlab\cite{gitlab}.

Cada nova ferramenta desenvolvida parte de uma ideia inicial básica e vai
amadurecendo através de reuniões presenciais ou remotas entre toda a equipe,
esta ideia inicial gera especificações em forma textual que posteriormente são
transformadas em desenhos, a partir destes desenhos com a proposta final da
funcionalidade a equipe de desenvolvimento da plataforma inicia seu trabalho e
entrega a funcionalidade tal qual foi desenhada anteriormente.

Este documento traz 2 novas propostas de ferramentas e contará com
especificação textual e modelos desenhos do que será desenvolvido
posteriormente pela equipe de desenvolvimento.

\subsection{Aplicativos para trilhas de participação}
 
O Participa.br utiliza algumas ferramentas como mecanismo de participação,
essas ferramentas permitem aos usuários da rede contribuir através de
comentários, votos ou diretamente sugerindo textos e conteúdos.

Estas ferramentas são organizadas em trilhas de participação, uma trilha
pode utilizar uma ou mais ferramentas e podem ter datas de início e fim de
participação, dando assim um fluxo para a participação social.

Nas próximas seções será apresentado 2 propostas de ferramentas ou aplicativos
para trilha de participação \index{cms}.

\subsubsection{Ferramenta para construção e debates de propostas}

Esta ferramenta é proposta inicialmente para atender a metodologia para a
consulta de regulamentação do Marco Civil, baseado em requisitos elaborados
pela equipe do Participa.br em conjunto com o Ministério da Justiça e será
desenvolvido como um plugin Noosfero.

A consulta de regulamentação do Marco Civil ocorrerá a partir de eixos de
debate, sem um texto base inicial, os eixos são:

\begin{enumerate}
  \item Neutralidade de rede
  \item Proteção de dados pessoais
  \item Registros de conexão e acesso a aplicações
  \item Fiscalização e apuração de infrações
  \item Políticas públicas para internet
\end{enumerate}

Para cada eixo de debate existirá uma breve explicação do que ele significa e
instruções básicas de como colaborar com ele, além de links para artigos
relacionados ao Marco Civil.

Os usuários interessados em contribuir para o debate poderão enviar propostas
para cada um dos eixos, estas propostas serão qualitativas e abertas. Os
usuários podem enviar quantas propostas quiserem. Cada proposta deverá seguir
o seguinte formato (ver Figura~\ref{fig:formulario-proposta}).

\begin{figure}[h]
\center
\includegraphics[scale=0.5]{04_-_proposta_passo_2.png}
\caption{Formulário de criação de propostas}
\label{fig:formulario-proposta}
\end{figure}

\begin{itemize}
  \item Título - 70 caracteres - para ajudar na divulgação nas redes sociais
  \item Resumo - 140 caracteres - para ajudar na divulgação nas redes sociais
%  \item Tags amplas pré-definidas com interface gameficada: forma de regulamentar, desafio, crítica, etc
  \item Forma de regulamentar - 1000 caracteres
% (campo customizado) (esta ferramenta deve ter suporte a campos personalizados, forma de regulamentar por exemplo é algo que só faz sentido nesta consulta do Marco Civil por exemplo)
\end{itemize}

%As propostas irão delimitar os eixos do debate. Propostas podem ser agregadas
%a partir da perspectiva do resumo. A forma de regulamentar será preservada e
%acompanhará o autor.

As propostas irão delimitar os eixos do debate e a discussão sobre cada tema
se dará através da interação dos usuários em cada proposta através de
comentários por parágrafo ou por trecho. Além claro da possibilidade de
concordar ou discordar da proposta como um todo.

Cada comentário poderá receber tags, estas tags darão semantica ao comentário
indicando inclusão ou remoção de texto. Desta forma além do próprio conteúdo
do comentário o usuário poderá também indicar se está sugerindo uma inclusão
ou remoção de texto daquela proposta.

Tuda vez que um usuário interage com uma proposta ele passa a acompanhar as
modificações daquela proposta através de seu perfil no Participa.br e pode a
qualquer momento voltar e contribuir com ela.

As propostas de cada eixo de debate formarão uma espécie de fórum de
discussão, mas com uma interface diferenciada e simples, com opções de
ordenação das propostas por vários critérios (ver
Figura~\ref{fig:pagina-inicial}). Exemplos de critérios de ordenação:

\begin{figure}[h]
\center
\includegraphics[scale=0.5]{01_-_pagina_inicial.png}
\caption{Proposta da interface da nova ferramenta, página inicial}
\label{fig:pagina-inicial}
\end{figure}

\begin{itemize}
  \item Ordenar de forma aleatória (padrão)
  \item Propostas mais discutidas
  \item Propostas com maior concordancia
  \item Propostas com menor concordancia
\end{itemize}

A interface de criação de propostas deve ser precedida por texto explicativo
de como funciona a consulta e a criação de propostas deve ser dividida em
passos pequenos e simples como pode ser visto das Figuras
\ref{fig:proposta02}, \ref{fig:proposta03} e \ref{fig:proposta04}.

\begin{figure}[h]
\center
\includegraphics[scale=0.5]{02_-_metodologia_e_eixos.png}
\caption{Proposta da interface da nova ferramenta, como funciona a consulta}
\label{fig:proposta02}
\end{figure}

\begin{figure}[h]
\center
\includegraphics[scale=0.5]{03_-_proposta_passo_1.png}
\caption{Proposta da interface da nova ferramenta, passo 1 da criacao de
propostas}
\label{fig:proposta03}
\end{figure}

\begin{figure}[h]
\center
\includegraphics[scale=0.5]{05_-_proposta_passo_final.png}
\caption{Proposta da interface da nova ferramenta, passo 3 da criacao de
propostas}
\label{fig:proposta04}
\end{figure}

Formas de brincar com essa consulta é ir muito além da mera visualização de
fórum mas permitir que os usuários filtrem dinamicamente os conteúdos em tempo
real e interajam com aqueles que mais lhes interessarem. Com isso, pode até
ser mais interessante não separar em 5 fóruns independentes para cada “eixo”
da consulta

\begin{figure}[h]
\center
\includegraphics[scale=0.4]{medium.png}
\caption{Proposta de comentários por parágrafo do Medium}
\label{fig:medium}
\end{figure}


-------

Deve ser possível ao administrador da consulta relacionar propostas dentro de
um mesmo subtema, de forma que facilite a discussão agregando propostas
similares e também o resultado final que será elaborado a partir desta
consulta. Propostas agregadas pelo administrador devem ser exibidas para os
usuários, ou sejam, os usuários devem conseguir identificar que sua proposta
foi anexada à outra.

Comentário por parágrafo o usuário pode comentar em parágrafos específicos de
um texto sugerindo alterações ou correções, ou mesmo tirando dúvidas.

\subsubsection{Votação em Pares}

A ferramenta de votação em pares permite grupos de pessoas coletar e priorizar
informações em um processo aberto, democrático e eficiente, esta ferramenta
deve combinar funcionalidades de coleta através de métodos quantitativos e
qualitativos. Deve ser escalável, rápido e permitir quantificar uma pesquisa
enquanto permite inclusão de novas informações pelos participantes.

Além de coletar e priorizar idéias, deve respeitar a privacidade do usuário de
forma que se evite qualquer possibilidade de identificação dos votos. De modo
geral a ferramenta funcionará da seguinte forma:

\begin{itemize}
  \item Administrador cria nova consulta e cadastra idéias para serem votadas
  \item Usuários da plataforma votam através da exibição em pares das
    idéias cadastradas conforme Figura \ref{fig:pairwise}
  \item O usuário pode cadastrar novas idéias, estas novas idéias entrarão
    para moderação
  \item Ao final do período de votação o administrador apura os resultados
    através da uma interface administrativa
\end{itemize}

As propostas de idéias inseridas pelo usuários devem ter limite de caracter e
a interface deve exibir um contador regressivo indicando quantos caracteres o
usuário já digitou, sugere-se um limite de 140 caracteres.

\begin{figure}[h]
\center
\includegraphics[scale=0.4]{design-votacao-em-pares-verde.png}
\caption{Proposta de interface de votação em pares}
\label{fig:pairwise}
\end{figure}

A implementação desta ferramenta deve gerar código HTML de incorporação em
ambientes externos. Além disso é imprescindível também ter uma interface
alternativa mais simples do mecanismo de votação para incorporação nas redes
sociais, esta interface alternativa não deve ter blocos laterais ou outros
elementos na página.

Para isso será importante aceitarmos votações de usuários não registrados para
os acessos oriundos desses códigos de incorporação externos. Para as votações
dentro do Participa.br, somente usuários cadastrados poderão votar.

Além dos critérios de usabilidade para usuário final é importante ter uma boa
interface de administração contemplando ao menos o seguinte:

\begin{itemize}
  \item Filtros para mediação das ideias cadastradas (ordenar por data, ordem
    alfabetica e nome)
  \item Novas ideias serão sempre mediadas
  \item Área de busca textual nas propostas cadastradas
  \item Busca por usuários cadastrados (para saber quais propostas
    um certo usuário cadastrou)
  \item Possibilidade de aprovação em massa através de checkbox
  \item Possiblidade de agregar duas propostas mantendo os dois autores
\end{itemize}

% http://noosfero.org/Development/ActionItem2942
% https://gitlab.com/participa/noosfero/tree/AI2942-pairwise_plugin/plugins/pairwise
% https://groups.google.com/forum/#!searchin/participa-dev/pairwise/participa-dev/iO5caqWf_FE/YYZm8SJw6DoJ
% https://gitlab.com/participa/noosfero/issues/47
% https://groups.google.com/forum/#!searchin/participa-dev/pairwise/participa-dev/G2dA6vt9C2M/ly3DtyF-Pa0J
% https://gitlab.com/participa/noosfero/issues/17
% https://groups.google.com/forum/#!searchin/participa-dev/pairwise/participa-dev/qyHpLqRsx84/y6i7K7GHCicJ
% https://gitlab.com/participa/noosfero/issues/97
% https://gitlab.com/participa/noosfero/issues/105

Esta ferramenta será implementada com base na API do pairwise, uma engine para
votação em pares, desenvolvido pela equipe do {\it Trustees of Princeton
University} disponível em:

\begin{itemize}
  \item https://github.com/allourideas/pairwise-api
\end{itemize}

O pairwise é implementado em {\it Ruby on Rails} e apesar de ser uma ótima
ferramenta possui uma séria limitação que impede seu uso no Participa.br,
a impossibilidade de trabalhar com banco de dados PostgreSQL, que é o banco de
dados já em uso no Participa.br.

Além disso será necessário adicionar novos métodos para filtrar e ordenar as
ideias cadastradas pois o pairwise não fornece tal possibilidade e isto é um
requisito necessário para o Participa.br.

% https://github.com/allourideas/pairwise-api/pull/30

O pairwise fornece toda infraestrutura para votação em pares, os dados são
armazenados e recuperados através dele, os algoritmos para cálculo de
relevancia dos votos, a lógica de negócio fica concentrada nele. Mas não
implementa nenhuma intertface, portanto será necessário implementar uma
interface para o pairwise no lado do Noosfero como plugin.

É importante destacar que o Noosfero carece de uma melhoria de performance nos
conteúdos gerados pelos plugins e será necessário otitimizar ao menos um
mecanismo de cache na aba de resultados exibidos pelo pairwise.

Esta interface irá se comunicar via API com uma instancia do pairwise
configurada nos servidores do Serpro, a documentação da API pode ser
encontrada em:

\begin{itemize}
  \item https://github.com/allourideas/pairwise-api/wiki/API-Documentation
\end{itemize}

\section{Conclusão}

Neste documento foi apresentado um \ProductDescription

Lembramos que para tornar o Portal de Consulta Pública realmente um canal de
consulta e participação popular na discussão e na definição da agenda
prioritária do país, é necessário que além de documentação faça-se um esforço
de movimentar as pessoar fora do ambiente virtual, para que haja um
engajamento no uso e contribuição deste projeto de forma consistente e perene.

\newpage
\bibliography{bibliografia}
\newpage
\listoffigures
\newpage
\printindex
\newpage
\definecolor{lightgrey}{rgb}{0.95,0.95,0.95}
\lstset{language=Ruby,basicstyle=\small\ttfamily,backgroundcolor=\color{lightgrey}}

\section{Anexos}

\subsection{Exemplo de código do plugin Noosfero para pairwise}

\lstinputlisting{pairwise_content.rb}

\subsection{Exemplo de código XML retornado pelo pairwise-api}

\begin{lstlisting}
<?xml version="1.0" encoding="UTF-8"?>
<prompt>
  <created-at type="datetime">2010-07-01T23:48:01+00:00</created-at>
  <id type="integer">1</id>
  <left-choice-id type="integer">10</left-choice-id>
  <question-id type="integer">7</question-id>
  <right-choice-id type="integer">9</right-choice-id>
  <tracking nil="true"></tracking>
  <updated-at type="datetime">2010-07-01T23:48:01+00:00</updated-at>
  <votes-count type="integer">0</votes-count>
  <left-choice-text>bar</left-choice-text>
  <right-choice-text>foo</right-choice-text>
</prompt>
\end{lstlisting}


%\appendix
%\appendixpage 
%\section{Foo bar}
\label{foobar}

%\lstinputlisting{observatorio.rb}


\end{document}
